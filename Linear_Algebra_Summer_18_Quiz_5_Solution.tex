\documentclass[answers,11pt]{exam}
\usepackage[utf8]{inputenc}
\usepackage[letterpaper,top=1.4in,bottom=1.2in,right=1.3in,left=1.3in]{geometry}

\usepackage{amsmath,amssymb}
%\usepackage{commath}
%\usepackage{mathtools}
%\usepackage{multicol}

\newcommand{\class}{Linear Algebra}
\newcommand{\term}{Summer 2018}
\newcommand{\examnum}{Quiz 5: Suggested Solutions}
\newcommand{\examdate}{08.06.18}
\newcommand{\timelimit}{20 minutes}

\pagestyle{head}
\firstpageheader{}{}{}
\runningheader{\class}{\examnum\ - Page \thepage\ of \numpages}{\examdate}
\runningheadrule

\usepackage{hyphenat}
\usepackage{graphicx}
\usepackage{tikz}
\usepackage{pgfplots}
\usepackage{mathrsfs}

\usepackage{array, booktabs, caption}
\usepackage{ragged2e}
\usepackage{multirow}
\usepackage{hhline}% http://ctan.org/pkg/hhline
\usepackage{makecell} 
\usepackage{enumitem}
\usepackage{bbm}
\usepackage{soul}

%\usepackage{mtpro2}
%\usepackage[nottoc]{tocbibind}

\usepackage[bottom]{footmisc}
\usepackage{natbib}
\usepackage{hyperref}
\usepackage{url}

\usepackage[normalem]{ulem}
\usepackage{relsize}
\usepackage{commath}
\usepackage{mathtools}
\allowdisplaybreaks

\usepackage{float}

\usepackage{color}
\usepackage{commath}
\usepackage{amsthm}

\newcommand\independent{\protect\mathpalette{\protect\independenT}{\perp}}
\def\independenT#1#2{\mathrel{\rlap{$#1#2$}\mkern2mu{#1#2}}}

\usepackage[utf8]{inputenc}
\usepackage[english]{babel}
\usepackage{textcomp}

\renewcommand\theadalign{lc}
\renewcommand\theadfont{\bfseries}
\usepackage{amssymb} %maths
\usepackage{amsmath} %maths

\usepackage[utf8]{inputenc} %useful to type directly diacritic characters

\usepackage{nomencl}
\makenomenclature

\usepackage{accents}

\makeatletter
\renewcommand*\env@matrix[1][*\c@MaxMatrixCols c]{%
	\hskip -\arraycolsep
	\let\@ifnextchar\new@ifnextchar
	\array{#1}}
\makeatother

\makeatletter
\renewcommand*\env@matrix[1][\arraystretch]{%
	\edef\arraystretch{#1}%
	\hskip -\arraycolsep
	\let\@ifnextchar\new@ifnextchar
	\array{*\c@MaxMatrixCols c}}
\makeatother


\newtheoremstyle{mytheoremstyle} % name
{\topsep}                    % Space above
{\topsep}                    % Space below
{}                   % Body font
{}                           % Indent amount
{\bfseries}                   % Theorem head font
{.}                          % Punctuation after theorem head
{.5em}                       % Space after theorem head
{}  % Theorem head spec (can be left empty, meaning ‘normal’)


\theoremstyle{definition}
\newtheorem{definition}{Definition} %[section]
\newtheorem{proposition}[definition]{Proposition}
\newtheorem{theorem}[definition]{Theorem}
\newtheorem{corollary}[definition]{Corollary}
\newtheorem{lemma}[definition]{Lemma}
\newtheorem{obs}[definition]{Observation}
\newtheorem{example}[definition]{Example}
\newtheorem{assumption}[definition]{Assumption}
\newtheorem{properties}[definition]{Properties}
\newtheorem{motivation}[definition]{Motivation}
\newtheorem{derivation}[definition]{Derivation}
\newtheorem{remark}[definition]{Remark}
\newtheorem{fact}[definition]{Fact}

\usepackage{caption}
\captionsetup[figure]{labelfont=sc}
\setlist[enumerate]{font=\bfseries\sffamily}



%\addto\captionsenglish{\renewcommand*{\proofname}{\scshape Proof.}}


%\numberwithin{equation}{section}

\DeclareMathOperator{\N}{\mathbb{N}}
\DeclareMathOperator{\Q}{\mathbb{Q}}
\DeclareMathOperator{\R}{\mathbb{R}}
\DeclareMathOperator{\Z}{\mathbb{Z}}
\DeclareMathOperator{\NN}{\mathcal{N}}
\DeclareMathOperator{\bindist}{\mathsf{B}}
\DeclareMathOperator{\DD}{D}
\DeclareMathOperator{\betadist}{Beta}
\DeclareMathOperator{\E}{\mathbf{E}}
\DeclareMathOperator{\cov}{Cov}
\DeclareMathOperator{\var}{Var}
\DeclareMathOperator{\samplespace}{\mathcal{S}}
\DeclareMathOperator{\suchthat}{\text{ s.t. }}
\DeclareMathOperator{\summod}{\accentset{\circ}{+}}
\DeclareMathOperator{\im}{im}
\DeclareMathOperator{\1}{\mathbbm{1}}
\DeclareMathOperator{\LL}{\mathscr{L}}
\DeclareMathOperator{\Imaginary}{Im}
\DeclareMathOperator{\supp}{Supp}
\DeclareMathOperator{\powerset}{\mathcal{P}}
\DeclareMathOperator{\normP}{norm}
\DeclareMathOperator{\BB}{\mathcal{B}}
\DeclareMathOperator{\contf}{\mathcal{C}}
\DeclareMathOperator{\riemannint}{\mathscr{R}}
\DeclareMathOperator{\osc}{osc}
\DeclareMathOperator{\sigmaalg}{\sigma-algebra}
\DeclareMathOperator{\MM}{\mathscr{M}}
\DeclareMathOperator{\M}{\mathcal{M}}
\DeclareMathOperator{\diam}{diam}
\DeclareMathOperator{\D}{\dif}
\DeclareMathOperator{\Span}{span}
\DeclareMathOperator{\PP}{\mathbf{P}}
\DeclareMathOperator{\CC}{\mathscr{C}}
\DeclareMathOperator{\sgn}{sgn}

\DeclareMathOperator{\col}{Col}
\DeclareMathOperator{\nul}{Null}

\renewcommand{\leq}{\leqslant}
\renewcommand{\geq}{\geqslant}
\renewcommand{\epsilon}{\varepsilon}
\renewcommand{\phi}{\varphi}

\newcommand{\Tau}{\mathcal{T}}
\newcommand{\zerov}{\mathbf{0}}

\newcommand{\uu}{\mathbf{u}}
\newcommand{\vv}{\mathbf{v}}

%\newcommand{\vect}[1][2]{\LL(#1,#2)}
\newcommand{\Lspace}[4]{\mathscr{L}^{#1}(#2,#3,#4)}
\newcommand{\condset}[4]{\left\{ #1  : \: #2 #3 #4 \right\}}
\newcommand{\ball}[2]{B(#1,#2)}
\newcommand{\innerproduct}[2]{\left\langle #1,#2 \right\rangle}
\newcommand{\polyn}[2]{\mathcal{P}_{#1}(#2)}
\newcommand{\GL}[2]{\mathrm{GL}_{#1}(#2)}
\newcommand{\proj}[2]{\mathrm{Proj}_{#1} {#2} }
\newcommand{\SL}[2]{\mathrm{SL}_{#1}(#2)}

%\titleformat{\section}
%{\centering\Large\normalfont\scshape}{\thesection .}{0.5em}{}
%
%\titleformat{\subsection}
%{\centering\large\normalfont\scshape}{\thesubsection .}{0.5em}{}

\renewcommand{\qedsymbol}{$\blacksquare$}


\begin{document}

\noindent
\begin{tabular*}{\textwidth}{l @{\extracolsep{\fill}} r @{\extracolsep{6pt}} l}
\textbf{\class} & \textbf{Name:} & \makebox[2in]{\hrulefill}\\
\textbf{\term} &&\\
\textbf{\examnum} &&\\
\textbf{\examdate} &&\\
\textbf{Time Limit: \timelimit} 
\end{tabular*}\\
\rule[2ex]{\textwidth}{2pt}

This quiz contains 2 sides (including this cover page) and 4 questions.\\
Total of points is \numpoints.


\begin{center}
Grade Table (for grader use only)\\
\addpoints
\gradetable[v][questions]
\end{center}

\noindent
\rule[2ex]{\textwidth}{2pt}

\begin{questions}

\question[30] Let $V$ be an inner product space with inner product $\langle\cdot,\cdot\rangle:V\times V\to \mathbb{R}$.
\noaddpoints
\begin{parts}
	\part[15] Define what it means for two vectors $v,w\in V$ to be orthogonal.
	
	\begin{solution}
		Two vectors $v,w \in V$ are \textbf{orthogonal} iff $\innerproduct{v}{w} = 0$.
		
		\textit{[10 for the statement above, 5 for completion. If some consequence of orthogonal vectors is given, then a maximum of 2 marks can be awarded]}
		
		\textit{[If the condition $v \cdot w = 0$ is provided (ie. mistaking the definition of orthogonal vectors in $V$ for that in $\R^n$), a maximum of 5 marks can be awarded]}
	\end{solution}
	
	\part[15] State the Gram-Schmidt Orthogonalization Algorithm for $\{v_1,\ldots, v_k\}\subseteq \mathbb{R}^n$. 10 bonus points if you specify it instead for a general inner product space $V$, i.e. $\{v_1,\ldots, v_k\}\subseteq V$.
	
	\begin{solution}
		For nontrivial subspaces of $\R^n$, given a basis $\set{v_1,v_2,\dots,v_k}$, Gram-Schmidt produces an orthogonal basis $\set{x_1,x_2,\dots,x_k}$ inductively, by setting
		\begin{align*}
			x_1 &= v_1 \\
			x_2 &= v_2 - \frac{ {x_1} \cdot {v_2}}{	{x_1} \cdot {x_1}} x_1 \\
			x_3 &= v_3 - \frac{ {x_1} \cdot {v_3}}{ {x_1} \cdot {x_1}} x_1 - \frac{ {x_2} \cdot {v_3}}{ {x_2} \cdot {x_2}} x_2 \\
			x_4 &= v_4 - \frac{ {x_1} \cdot {v_4}}{ {x_1} \cdot {x_1}} x_1 - \frac{ {x_2} \cdot {v_4}}{ {x_2} \cdot {x_2}} x_2 - \frac{ {x_3} \cdot {v_4}}{ {x_3} \cdot {x_3}} x_3 \\
			\vdots & \phantom{==} \vdots \\
			x_k &= v_k - \left( \sum_{i=1}^{k-1} \frac{ {x_i} \cdot {v_k} }{ {x_i} \cdot {x_i}} x_i \right)
		\end{align*}
		
		For a general inner product space, Gram-Schmidt works similarly---just replacing dot products with inner products:
		\begin{align*}
			x_1 &= v_1 \\
			x_2 &= v_2 - \frac{\innerproduct{x_1}{v_2}}{\innerproduct{x_1}{x_1}} x_1 \\
			x_3 &= v_3 - \frac{\innerproduct{x_1}{v_3}}{\innerproduct{x_1}{x_1}} x_1 - \frac{\innerproduct{x_2}{v_3}}{\innerproduct{x_2}{x_2}} x_2 \\
			x_4 &= v_4 - \frac{\innerproduct{x_1}{v_4}}{\innerproduct{x_1}{x_1}} x_1 - \frac{\innerproduct{x_2}{v_4}}{\innerproduct{x_2}{x_2}} x_2 - \frac{\innerproduct{x_3}{v_4}}{\innerproduct{x_3}{x_3}} x_3 \\
			\vdots & \phantom{==} \vdots \\
			x_k &= v_k - \left( \sum_{i=1}^{k-1} \frac{ \innerproduct{x_i}{v_k} }{\innerproduct{x_i}{x_i}} x_i \right)
		\end{align*}
		
		In either case, we know that $\Span\set{v_1,v_2,\dots,v_k} = \Span\set{x_1,x_2,\dots,x_k}$, by a theorem introduced in class.
		
		\textit{[5 for stating \uline{correctly} $\proj{\Span \set{ x_j}_{j=1}^{i-1} }{v_i}$ for $i \geq 2$, 8 for correctly stating the algorithm for basis $\set{v_1,\dots,v_k}$, 2 for having all of the above. For the extra credit, 2 for recognising to switch notations, 6 for correct restatements of algorithm, 2 for successful completion.]}
	\end{solution}
\end{parts}

\addpoints


\question[20] Given 
\begin{equation*}
x_1=
\begin{bmatrix}
1\\
1\\
1
\end{bmatrix}\:\:\:
x_2=
\begin{bmatrix}
0\\
1\\
1
\end{bmatrix}\:\:\:
x_3=
\begin{bmatrix}
0\\
0\\
1
\end{bmatrix}
\end{equation*}
compute an \textbf{orthonormal} basis for $W=\text{span}(x_1,x_2,x_3)$. 
\addpoints

\begin{solution}
	Notice that $W = \R^3$ because $\set{x_1,x_2,x_3}$ is a linearly independent set of vectors in $\R^3$, hence it spans $\R^3$. The canonical/standard basis in $\R^3$ suffices.
	
	Gram-Schmidt also works as well.
	
	\textit{[5 for recognising that $W = \R^3$, 5 for two correct orthonormal vectors in the new basis, 5 for the third correct vector, 5 for having all of the above]}
	
	\textit{[If one uses G-S, then: 5 for attempting to use G-S, 5 for two correct orthonormal vectors in the new basis, 5 for the third correct vector, 5 for having all of the above]}
\end{solution}

\question[25] Compute the orthogonal projection of $y$ onto $W = \text{span}(u_1,u_2)$ where 
\begin{equation*}
y=
\begin{bmatrix}
6\\
3\\
-2
\end{bmatrix}\:\:\:
u_1=
\begin{bmatrix}
3\\
-1\\
2
\end{bmatrix}\:\:\:
u_2=
\begin{bmatrix}
1\\
-1\\
-2
\end{bmatrix}.
\end{equation*}
\addpoints

\begin{solution}
	First, notice that $\innerproduct{u_1}{u_2}_{\R^3} = 0$, implying they are linearly independent. This spans a plane in $\R^3$. We now know that the question makes sense. To calculate the projection, and with the knowledge that $\innerproduct{\cdot}{\cdot}$ in the calculations below mean the inner product in $\R^3$, we have
	\begin{align}
	\nonumber	\proj{W}{y} &= \frac{\innerproduct{y}{u_1}}{\innerproduct{u_1}{u_1}} u_1 + \frac{\innerproduct{y}{u_2}}{\innerproduct{u_2}{u_2}} u_2 \\
	\label{eq1}	&= \frac{11}{14} v_1 + \frac{7}{6} v_2\\
	\label{eq2}	&= \begin{bmatrix}[1.2]
		\frac{74}{21} \\ -\frac{41}{21} \\ -\frac{16}{21}
		\end{bmatrix}
	\end{align} 
	
	\textit{[5 for checking $\set{u_1,u_2}$ is orthogonal, 5 for subsequently concluding that $\set{u_1,u_2}$ is an orthogonal basis, 2 for stating/acknowledging \uline{correctly} the projection formula, 2 for attempt at calculation, 4 for obtaining \eqref{eq1}, 4 for obtaining \eqref{eq2}, 3 for having all of the above]}

	\textit{[If one does not justify $\set{u_1,u_2}$ is an orthogonal basis before using the formula, a maximum of 10 marks is awarded]}
\end{solution}

\question[25] Given the inner product on $C^0([0,1],\mathbb{R})$ defined by 
\begin{equation*}
\langle f,g\rangle =\int_0^1 f(t)g(t)dt
\end{equation*}
determine whether or not $p(t)=2t-1$ and $q(t)=10t$ are orthogonal. (Show work)

\begin{solution}
	A simple calculation gives
	\begin{align*}
		\innerproduct{p}{q} &= \int_{0}^{1} 10t(2t - 1) \D t = \int_{0}^{1} 20t^2 - 10t \D t \\
		&= \eval{\left[ \frac{20}{3} t^3 - 5t^2 \right]}_{0}^{1} \\
		&= \frac{5}{3}
	\end{align*}
	hence the functions are not orthogonal.
	
	\textit{[2 for any attempt, 3 for \uline{correctly} setting up the integral, 5 for \uline{correct} integrand, 8 for \uline{correct} calculations, 2 for \uline{correct} evaluation of the integral, 5 for subsequent conclusion]}
\end{solution}


\end{questions}

\end{document}

