\documentclass{amsart}
%\documentclass[11pt]{article}
%\usepackage{amscls}
\linespread{1}
%\usepackage[letterpaper,top=1.2in,bottom=1in,right=1.03in,left=1.03in]{geometry}
\usepackage[letterpaper,top=1.6in,bottom=1.3in,right=1.3in,left=1.3in]{geometry}
\usepackage{titlesec}
\usepackage{lipsum}

%\usepackage{libertine}
%\usepackage[lite,subscriptcorrection,nofontinfo,amsbb,eucal]{mtpro2}
%slantedGreek

\usepackage{hyphenat}
\usepackage{graphicx}
\usepackage{tikz}
\usepackage{pgfplots}
\usepackage{mathrsfs}



\usepackage{array, booktabs, caption}
\usepackage{ragged2e}
\usepackage{multirow}
\usepackage{hhline}% http://ctan.org/pkg/hhline
\usepackage{makecell} 
\usepackage{enumitem}
\usepackage{bbm}
\usepackage{soul}

%\usepackage{mtpro2}


%\usepackage[nottoc]{tocbibind}
\usepackage[bottom]{footmisc}
\usepackage{natbib}
\usepackage{hyperref}
\usepackage{url}

\usepackage[normalem]{ulem}
\usepackage{relsize}
\usepackage{commath}
\usepackage{mathtools}
\allowdisplaybreaks

\title{\textsc{Assignment 1 : Chapter 1}}
\thanks{Summer 2018. Instructor: Antonios-Alexandros Robotis.}
\date{\today}


\usepackage{float}

\usepackage{color}
\usepackage{commath}
\usepackage{amsthm}

\newcommand\independent{\protect\mathpalette{\protect\independenT}{\perp}}
\def\independenT#1#2{\mathrel{\rlap{$#1#2$}\mkern2mu{#1#2}}}

\usepackage[utf8]{inputenc}
\usepackage[english]{babel}
\usepackage{textcomp}

\renewcommand\theadalign{lc}
\renewcommand\theadfont{\bfseries}
\usepackage{amssymb} %maths
\usepackage{amsmath} %maths

\usepackage[utf8]{inputenc} %useful to type directly diacritic characters

\usepackage{nomencl}
\makenomenclature

\usepackage{accents}

\makeatletter
\renewcommand*\env@matrix[1][*\c@MaxMatrixCols c]{%
	\hskip -\arraycolsep
	\let\@ifnextchar\new@ifnextchar
	\array{#1}}
\makeatother


\usepackage{fancyhdr}

\pagestyle{fancy}
\fancyhf{}
\renewcommand{\headrulewidth}{0.5pt}
\rhead{\textsc{Assignment 1}}
\lhead{\textsc{Linear Algebra}}
%\lhead{\textsc{}}
\cfoot{\thepage}

\usepackage{lmodern}

\newtheoremstyle{mytheoremstyle} % name
{\topsep}                    % Space above
{\topsep}                    % Space below
{}                   % Body font
{}                           % Indent amount
{\bfseries}                   % Theorem head font
{.}                          % Punctuation after theorem head
{.5em}                       % Space after theorem head
{}  % Theorem head spec (can be left empty, meaning ‘normal’)


%\theoremstyle{mytheoremstyle}
%
\theoremstyle{definition}
\newtheorem{definition}{Definition}%[section]
\newtheorem{proposition}[definition]{Proposition}
\newtheorem{theorem}[definition]{Theorem}
\newtheorem{corollary}[definition]{Corollary}
\newtheorem{lemma}[definition]{Lemma}
\newtheorem{obs}[definition]{Observation}
\newtheorem{example}[definition]{Example}
\newtheorem{assumption}[definition]{Assumption}
\newtheorem{properties}[definition]{Properties}
\newtheorem{motivation}[definition]{Motivation}
\newtheorem{derivation}[definition]{Derivation}
\newtheorem{remark}[definition]{Remark}
\newtheorem{fact}[definition]{Fact}

\theoremstyle{definition}
\newtheorem*{solution}{Solution}

\usepackage{caption}
\captionsetup[figure]{labelfont=sc}
\setlist[enumerate]{font=\bfseries\sffamily}



%\addto\captionsenglish{\renewcommand*{\proofname}{\scshape Proof.}}


%\numberwithin{equation}{section}

\DeclareMathOperator{\N}{\mathbb{N}}
\DeclareMathOperator{\Q}{\mathbb{Q}}
\DeclareMathOperator{\R}{\mathbb{R}}
\DeclareMathOperator{\Z}{\mathbb{Z}}
\DeclareMathOperator{\NN}{\mathcal{N}}
\DeclareMathOperator{\bindist}{\mathsf{B}}
\DeclareMathOperator{\DD}{D}
\DeclareMathOperator{\betadist}{Beta}
\DeclareMathOperator{\E}{\mathbf{E}}
\DeclareMathOperator{\cov}{Cov}
\DeclareMathOperator{\var}{Var}
\DeclareMathOperator{\samplespace}{\mathcal{S}}
\DeclareMathOperator{\suchthat}{\text{ s.t. }}
\DeclareMathOperator{\summod}{\accentset{\circ}{+}}
\DeclareMathOperator{\im}{im}
\DeclareMathOperator{\1}{\mathbbm{1}}
\DeclareMathOperator{\LL}{\mathscr{L}}
\DeclareMathOperator{\Imaginary}{Im}
\DeclareMathOperator{\supp}{Supp}
\DeclareMathOperator{\powerset}{\mathcal{P}}
\DeclareMathOperator{\normP}{norm}
\DeclareMathOperator{\BB}{\mathscr{B}}
\DeclareMathOperator{\contf}{\mathcal{C}}
\DeclareMathOperator{\riemannint}{\mathscr{R}}
\DeclareMathOperator{\osc}{osc}
\DeclareMathOperator{\sigmaalg}{\sigma-algebra}
\DeclareMathOperator{\MM}{\mathscr{M}}
\DeclareMathOperator{\diam}{diam}
\DeclareMathOperator{\D}{\dif}
\DeclareMathOperator{\Span}{span}
\DeclareMathOperator{\PP}{\mathbf{P}}
\DeclareMathOperator{\CC}{\mathscr{C}}
\DeclareMathOperator{\sgn}{sgn}

\renewcommand{\leq}{\leqslant}
\renewcommand{\geq}{\geqslant}
\renewcommand{\epsilon}{\varepsilon}
\renewcommand{\phi}{\varphi}

\newcommand{\Tau}{\mathcal{T}}

%\newcommand{\vect}[1][2]{\LL(#1,#2)}
\newcommand{\Lspace}[4]{\mathscr{L}^{#1}(#2,#3,#4)}
\newcommand{\condset}[4]{\left\{ #1  : \: #2 #3 #4 \right\}}
\newcommand{\ball}[2]{B(#1,#2)}
\newcommand{\innerproduct}[2]{\left\langle #1,#2 \right\rangle}


%\titleformat{\section}
%{\centering\Large\normalfont\scshape}{\thesection .}{0.5em}{}
%
%\titleformat{\subsection}
%{\centering\large\normalfont\scshape}{\thesubsection .}{0.5em}{}

\renewcommand{\qedsymbol}{$\blacksquare$}

\begin{document}
\sloppy
\maketitle

%In what follows, let
%\begin{itemize}[itemsep=0mm]
%	\item $V$ be an inner product space.
%\end{itemize}

This document contains suggested solutions to a selection of \textit{representative} problems (which are graded) from the current homework assignment---one from each subsection of chapter one in the textbook. They do not include every last detail, but should give you enough guidance to complete each problem fully on your own. 

Even though the rest of the homework assignment are checked for completion, the grader will make an attempt at pointing out glaring faults in arguments whenever they arise. As such, some general comments on the completion of this assignment are provided at the end. 

Below are a few conventions are used throughout grading your assignments, along with some ground rules for grading:


\begin{enumerate}[itemsep=.75em]
	
%	\item Raw score is given /100, but normalised to /10 on NYUClasses. The raw score makes partial marks easier to give out.
	
	\item Up to 40 raw marks are given for completion of homework. The actual score given depends on the amount of assignment completed, and quality of the work attempted. This will be given in 5-mark increments. As a rough description for this raw score, the table below describes how this score will be assigned:
	
	\begin{center}
		\begin{tabular}{c | p{9cm}}
			\textbf{Mark(s)} & \multicolumn{1}{c}{\textbf{Description}} \\
			\hline
			0 & Did not complete anything in the assignment / did not hand it in. \\
			\hline
			10 & A lighthearted attempt at a few questions are made. \\
			\hline
			20 & Up to half of the questions have been attempted, with some effort put in. \\
			\hline
			30 & More than three-fourths of the questions have been attempted, all with considerable progress. \\
			\hline
%			40 & Missing attempts on only a few questions in the assignment. \\
%			\hline
			40 & Attempted every question assigned, and has made considerable progress in every question.
		\end{tabular}
	\end{center}
	
%	\item When grading the questions indicated below, the following scale and description are considered:
%	
%	\begin{center}
%		\begin{tabular}{c | c}
%			\textbf{Mark(s)} & \textbf{Description} \\
%			\hline
%			0 & Did not complete anything to solve the problem. \\
%			2 & A mostly incorrect attempt has been made. \\
%			4 & A somewhat correct attempt has been made. \\
%			6 & One to a few mistakes has been made. \\
%			8 & One to two mistakes has been made, with wrong calculations or conclusions. \\
%			10 & Looks great!
%		\end{tabular}
%	\end{center}
%	
%	\bigskip
%	
%	\noindent Any odd-value scores are given at the discretion of the grader.
	
	\item Marks available and breakdown are indicated at the beginning of every question, inside square brackets. 
	
	\item Naming convention: in the following section, question indexed as $1.x.y$ should be read as ``question $y$ from chapter $1.x$ in Lay's textbook''.
	
	\item If you wish to dispute grading on question(s), please hand in your assignment to the instructor on the next homework submission day---with your comments/disputes written around the question(s).
	
\end{enumerate}



\clearpage

\section*{Suggested Solutions}

\bigskip

\begin{enumerate}[itemsep = 2mm]
	\item[1.1.25] \textit{[\textbf{4}; 1 for attempt at row reduction, 2 for correct row reduction, 1 for correct equation]}
	
	
	Row reductions read
	\begin{align*}
	\begin{bmatrix}[c c c | c]
	1 & -4 & 7 & g \\
	 & 3 & -5 & h \\
	-2 & 5 & -9 & k
	\end{bmatrix} &\leadsto \begin{bmatrix}[c c c | c]
	1 & -4 & 7 & g \\
	& 3 & -5 & h \\
	 & -3 & 5 & k + 2g
	\end{bmatrix} \\
	&\leadsto \begin{bmatrix}[c c c | c]
	1 & -4 & 7 & g \\
	& 3 & -5 & h \\
	&  &  & k + 2g + h
	\end{bmatrix} 
	\end{align*}
	hence, for the system to be consistent (ie. make sense to solve), we need $k + 2g + h = 0$.
	
	
	\item[1.2.30] \textit{[\textbf{4}; 1 for an attempt at providing an example, 1 for a correct example, 2 for justification]}
	
	A dumb example is the following:
	\begin{align*}
	\begin{cases}
		x + 2y + 3z &= 2 \\
		0 &= 5
	\end{cases}
	\end{align*}
	A slightly more ``reasonable'' inconsistent system is the following:
	\begin{align*}
	\begin{cases}
	x + 2y + 3z &= 2 \\
	2x + 4y + 6z &= 2 \\
	\end{cases}
	\end{align*}
	One needs to justify the example results in an inconsistent system---by row reduction.
	
	\item[1.3.24] \textit{[\textbf{10}; 2 for each part---1 for T/F, 1 for correct justification]}
	
	\begin{enumerate}
		\item \textbf{True}. That is a remark made early on in the subsection.
		
		\item \textbf{True}. Formally,
		\begin{align*}
		v + (u-v) &= (v + u) + (-v) & \text{associativity}\\
		&= u +v + (-v) & \text{commutativity} \\
		&= u & \text{additive inverse}
		\end{align*}
		This will make more sense once the definition of vector space is given.
		
		\item \textbf{False}. The counterexample? If $\set{v_1,v_2,\dots,v_p}$ is a collection of linearly independent vectors.
		
		\item \textbf{True}. The span contains all linear combinations of $u$ and $v$.
		
		\item \textbf{True}. By definition.
	\end{enumerate}
	
	
	
	
	\item[1.4.30] \textit{[\textbf{6}; 1 for an attempt at the question, 2 for a correct example, 3 for justification]}
	
	Any $3 \times 3 $ matrix $A$ of the form
	\begin{align*}
	A = \begin{bmatrix}
	| & | & | \\
	a_1 & a_2 & a_3 \\
	| & | & | 
	\end{bmatrix}
	\end{align*}
	with two linearly dependent columns will not span the entire $\R^3$ (or, put it another way: any $3 \times 3 $ matrix \textit{without} 3 linearly independent columns does not span the entire $\R^3$), hence we can always find a vector $v \in \R^3$ that does not lie in $\Span \set{a_1,a_2,a_3}$. One needs to show such an $A$, along with a vector $b \in \R^3$, such that the reduced form of the augmented matrix
	\begin{align*}
	\begin{bmatrix}[ccc|c]
	| & | & | & |\\
	a_1 & a_2 & a_3  & b \\
	| & | & |  & |
	\end{bmatrix}
	\end{align*}
	results in a contradiction, ie. last row reads $0x+0y+0z = k$, where $k \in \R \setminus \set{0}$. To truly justify your answer, you will need to show (\textit{correctly}) that the row-reduced form presents a contradiction.
	
	
	\item[1.5.24] \textit{[\textbf{10}; 2 for each part---1 for T/F, 1 for correct justification]}
	
	\begin{enumerate}
		\item \textbf{False}. For all matrix equations $Ax = 0$ and $x \neq \mathbf{0}$, $x$ can contain some zero entries while the rest are nonzero, and still be considered as nontrivial solution.
		
		\item \textbf{True}. That almost follows from definition: two vectors are linearly independent if one is not a scalar multiple of the other. Hence, $x_2 u + x_3 v = 0$ admits a solution.
		
		
		\item \textbf{True}. For all matrices $A$ with $A0 = b$, this implies $b = 0$!
		
		\item \textbf{True}. We can also think of adding $p$ as sliding the vector along the direction given by $p$. Draw a diagram in $\R^2$ to see what is happening.
		
		\item \textbf{False}. In general, this is not true; by Theorem 6 in Lay, $Ax =b$ would need to be consistent, and we are not told to assume this.
	\end{enumerate}
	
	
	
	\item[1.7.21] \textit{[\textbf{8}; 2 for each part---1 for T/F, 1 for correct justification]}
	
	\begin{enumerate}
		\item \textbf{False}. The trivial solution is always a solution.
		
		\item \textbf{False}. As an counterexample,
		\begin{align*}
		\begin{bmatrix} 1 \\ 2 \end{bmatrix}, \begin{bmatrix} 2 \\ 4 \end{bmatrix}, \begin{bmatrix} 0 \\ -7 \end{bmatrix}
		\end{align*}
		is a set of linearly dependent vectors, but the last is not a linear combination of the first two.
		
		\item \textbf{True}. Look at Theorem 8 in Lay.
		
		
		\item \textbf{True}. Since $x$ and $y$ are linearly independent, and $\set{x,y,z}$ is linearly dependent, it must be the case that $z$ can be written as a linear combination of the other two, thus $z \in \Span \set{x,y}$.
	\end{enumerate}
	
	
	
	\item[1.8.35] \textit{[\textbf{8}; 2 for stating what needs to be proven, 6 for the intermediary steps and conclusion. If only an example is provided, a maximum of 1 is given]}
	
	\begin{proof}
		To show that the reflection map is linear, it suffices to show that both linearity and scalar multiplicative properties hold. Let $u,v \in \R^3$ and $a,b \in \R$. Then, it follows that
		\begin{align*}
		T(au + bv) &= (a u_1 + b v_1, a u_2 + b v_2, - (a u_3 + b v_3) ) \\
		&= (a u_1 + b v_1, a u_2 + b v_2, - a u_3 - b v_3) \\
		&= (a u_1, a u_2, - a u_3) + (b v_1, b v_2, - b v_3) \\
		&= a (u_1, u_2, - u_3) + b (v_1, v_2, -v_3) \\
		&= a T(u) + b T(v)
		\end{align*}
		as desired (along the way, justify steps using linearity of vector additions).
	\end{proof}
	
	
	
	\item[1.9.23] \textit{[\textbf{10}; 2 for each part---1 for T/F, 1 for correct justification]}
	
	\begin{enumerate}
		\item \textbf{True}. Let the \textit{canonical basis\footnote{\textit{Basis} is to be defined later in this course.} in $\R^n$}, denoted $(e_1,e_2,\dots,e_n)$, be a collection of column vectors where
		\begin{align*}
		e_i = (0,0,\dots,\underbrace{1}_{i\text{-th position}},\dots, 0)
		\end{align*}
		for all $1 \leq i \leq n$. The columns on the identity matrix of size $n$ are the vectors in this canonical vectors. Since every vector can be written as a linear combination of these vectors, and $T$ is a linear transformation, if we know the image of these vectors, we know everything about the transformation.
		
		
		\item \textbf{True}. One needs to show that properties of linear transformation are preserved under rotation; in particular, we need to show that rotation preserves norm and angle. The presentation of a formal proof is not needed to get full marks on this; one just needs to state that linear transformation is preserved under rotation. For those who are interested, here is the proof.
		
		% We can show this by first showing that rotations preserve dot products, then show that any map that preserves dot products are necessarily linear. Then, we show the map is scalar multiplicative. This concludes the proof.
		
		\bigskip
		
		\begin{proof}
			For any angle $\phi$, let $T_{\phi} : \R^2 \mapsto \R^2$ be defined as
			\begin{align*}
			T_\phi (x,y) = \begin{cases}
				\text{counterclockwise rotation by } \phi & \text{ if } (x,y) \neq 0 \\
				(0,0) & \text{ if } (x,y) = (0,0)
			\end{cases}
			\end{align*}
			Now we find the explicit formula of $T_\phi$. Let $\alpha$ be the angle that $(x,y)$ makes with the positive $x$-axis. Let $r = \sqrt{x^2 + y^2}$. Then, $x = r \cos \alpha$ and $y = r \sin \alpha$. In addition, $T_\phi (x,y) $ has length $r$ and makes angle $\alpha + \phi$ with the positive $x$-axis. It follows that
			\begin{align*}
			T_\phi (x,y) &= (r\cos(\alpha + \phi), r\sin(\alpha + \phi)) & \\
			&= (r[\cos \alpha \cos \phi - \sin \alpha \sin \phi], r[\cos \alpha \sin \phi + \sin \alpha \cos \phi]) & (\text{compound angle formulae}) \\
			&= (x \cos \phi - y \sin \phi, x \sin \phi + y \cos \phi) & (\text{substitution})
			\end{align*}
			
			To show this is indeed a linear map, let $x \coloneqq (a_1,a_2)$ and $y \coloneqq (b_1,b_2)$, and $c,d \in \R$. Then, noting that
			\begin{align*}
			cx + dy = c(a_1,a_2) + d(b_1,b_2) = (c a_1 + d b_1, c a_2 + d b_2)
			\end{align*}
			we then have
			\begin{align*}
			T_\phi (cx + dy) &= ( (c a_1 + d b_1) \cos \phi - (c a_2 + d b_2) \sin \phi, (c a_1 + d b_1) \sin \phi + (c a_2 + d b_2) \cos \phi ) \\
			&= ( c(a_1 \cos \phi - a_2 \sin \phi) + d(b_1 \cos \phi - b_2 \sin \phi), d(b_1 \cos \phi + b_2 \sin \phi) + c (a_1 \sin \phi + a_2 \cos \phi) ) \\
			&= c(a_1 \cos \phi - a_2 \sin \phi, a_1 \sin \phi + a_2 \cos \phi) + d(b_1 \cos \phi - b_2 \sin \phi, b_1 \sin \phi + b_2 \cos \phi) \\
			&= c T_\phi(x) + d T_\phi (y)
			\end{align*}
			as desired.
		\end{proof}
		
		\item \textbf{False}. Let $T$ and $S$ be linear transformations; they are both closed under addition and scalar multiplication. Check the properties to see that the result is always a linear transformation, ie. $T \circ S$ (or $S \circ T$) are both linear transformations.
		
		\item \textbf{False}. The statement here is exactly the other way around.
		
		\item \textbf{False}. If the matrix has two pivots, then it is injective (one-to-one) but not surjective (onto).
	\end{enumerate}
	
	
	\item[Extra Credit] \textit{[\textbf{10}; (a): 2, (b): 3, (c): 2, (d): 3]}
	
	
	Parts (a) through (d) can be summarised in two separate propositions. Under the following qualifiers:
	
	Let $E_1,\ldots, E_m$ denote a collection of linear equations, where, for $a_{ij}, b_j\in \mathbb{R}$ for $1\le i\le n$ and $1\le j\le m$, we have
	\begin{equation*}
	\begin{matrix}
	E_1 &:& a_{11}x_1+\cdots+ a_{1n}x_n=b_1\\
	\vdots& & \vdots \\
	E_m &:& a_{m1}x_1+\cdots+ a_{mn} x_n=b_m
	\end{matrix}
	\end{equation*}
	
	
	\begin{proposition}
		\label{prop1}
		$(s_1,\ldots, s_n)$ is a solution to $(E_1,\dots,E_m)$ iff it is also a solution to $(E_1,\dots,k E_j,\dots,E_m)$, for any nonzero $k\in \mathbb{R}$ and $1 \leq j \leq m$.
	\end{proposition}
	
	\begin{proposition}
		\label{prop2}
		$(s_1,\ldots, s_n)$ is a solution to $(E_1,\dots,E_m)$ iff it is also a solution to $(E_1, \dots, E_{j-1}, E_j + k E_\ell, \dots, E_m)$, for any $k\in \mathbb{R}$ and $1 \leq j,\ell \leq m$.
	\end{proposition}
	
	
	\begin{proof}
		Note that in order to prove \propref{prop1} and \propref{prop2}, we need to prove that the solution sets to matrix equations $Ax = b$ and $\overline{A} x = b$, for $A$ and $\overline{A}$ (after performing elementary row operations on $A$), are the same. Formally, let $S_1$ be the solution set to $Ax = b$ and $S_2$ be that to $\overline{A} x = b$. It remains to show that $S_1 = S_2$, and we do that by showing both $S_1 \subset S_2$ and $S_1 \supset S_2$.
		
		\uline{$S_1 \subset S_2$} (ie. any solution to the original system is a solution for the new system). This is almost trivially true, but takes some verification.
		\begin{enumerate}
			\item Suppose $A \xrightarrow{(i) \leftrightarrow (j)} \overline{A}$. We know solution sets are not dependent on the order of the equations, hence $S_1 \subset S_2$.
			
			\item Suppose $A \xrightarrow{k \times (i)} \overline{A}$. We know, for row $i$, if $(s_1,s_2,\dots,s_n)$ is a solution, then $a_{i1} s_1 + a_{i2} s_2 + \dots + a_{in} s_n = b_i$, and $k$ multiples of this row is $k a_{i1} s_1 + k a_{i2} s_2 + \dots + k a_{in} s_n = k b_i$. For any $k \in \R \setminus \set{0}$, the multiplicative inverse exists, ie. $\frac{1}{k}$ exists and is a real number. Clearly, $(s_1,s_2,\dots,s_n)$ is still a solution to the new system. 
			
			\item Suppose $A \xrightarrow{(i) + k \times (j)} \overline{A}$. Then, row $i$ of $\overline{A}$ reads $(a_{i1} + k a_{j1}) s_1 + \dots + (a_{in} + k a_{jn}) s_n = b_i + k b_j$, where it is obvious that $(s_1,s_2,\dots,s_n)$ is still a solution to the system.
		\end{enumerate}
		We conclude that $S_1 \subset S_2$.
		
		\uline{$S_1 \supset S_2$}. We rely on the fact that elementary row operations admit inverses: this is clear for row swaps and row additions, and for scalar multiplication of a row, this is only possible for all $k \in \R \setminus \set{0}$ (so to guarantee the existence of the multiplicative inverse). The argument is very similar to that above, so we omit it here.

		Combining the two cases above, we have $S_1 = S_2$, as desired.
	\end{proof}
	
	
\end{enumerate}


\clearpage


\section*{General Comments}

The point of studying mathematics is to be able to come up with \textit{simple, nontrivial examples}, from which one can think about the fundamental principles underlying them. Some of these T/F questions help one in doing exactly that. It is important to understand how counterexamples arise in mathematics, hence I grade them in spades in this chapter. As is also remarked in the general comments to Quiz 1, it is important to know what mathematical objects we are dealing with---hence the proofs-type questions are useful in assessing how comfortable you are with these objects.

\textbf{If a student wants to have part(s) of assignment 1 regraded/rechecked}, have him/her hand in the assignment next Monday. Have him/her circle the question of interest, let them write a justification for regrading, and I'll see to it.

Here are, again, some general comments:

\begin{enumerate}[itemsep = 1.5mm]
	\item[1.1.25] \uline{Check the numbers during row reductions carefully}.
	
	
	\item[1.2.30] \uline{Justify the example you gave}. Why does the matrix you give not have solutions / is inconsistent? Row reductions should be shown.
	
	\item[1.3.24] \begin{itemize}
		\item \uline{Justify the choice you made}: if it is true, state why it is generally true; if it is false, come up with a \textit{counterexample}. One can also justify why the statement is false by pulling out some theorems, but principle of parsimony is key: a simple, nontrivial counterexample should be used whenever possible.
		
		\item \uline{Use properties of some operation whenever necessary}: use the properties of vector addition in part (b) is necessary, for example. Simply stating it would suffice.
		
		\item \uline{Use definition whenever necessary}: if a statement is true/false by definition, state exactly that.
	\end{itemize}

	
	\item[1.4.30] \begin{itemize}
		\item \uline{Provide the full example}: it is good if you justified how a matrix's columns do not span $\R^3$, but please then find such a vector that is not in the span of the column vectors!
		
		\item \uline{Full generality}: some chose to take some arbitrary vector $\vec{b} \coloneqq (b_1,b_2,b_3)$, and found the necessary condition for an augmented matrix $[A | b]$ (the matrix $A$ is provided by the student) to be inconsistent. While the solution is in its full generality, the question did \textit{not} ask for such generality; I have accepted this type of answers this time round (given it is done correctly), because it is objectively a better answer. We are often interested in the general conditions that induces a solution to a system, so if you went the extra mile, I have reciprocated by accepting your effort and answer.
		
		\item \uline{Lack of $\vec{b}$}: in tandem with the first remark, I took off 1 mark for people who did not provide such a vector $\vec{b}$.
	\end{itemize}
	
	
	\item[1.5.24] \begin{itemize}
		\item \uline{Justify your responses}.
		
		\item \uline{Drawing diagram as justification}. Generally, this is not a great way to justify a response; refrain from this (especially in a \textit{proof}) in the future.
	\end{itemize}
	
	
	
	\item[1.7.21] \begin{itemize}
		\item \uline{Justify your responses}.
		
		\item \uline{Part a}: regardless of $A$, the matrix equation $A \vec{x} = \vec{0}$ \textit{always} admits the trivial solution! The trivial solution is 
		
		\item \uline{Part b}: if false, it is preferable to use a counterexample. Principle of parsimony applies: if a counterexample can invalidate the statement, then it is more preferable than a theorem. I was harsh on this point this time, but, in the future, I will try my best to read through you invoking some theorem from textbook.
	\end{itemize}
	
	
	
	\item[1.8.35] \begin{itemize}
		\item \uline{Understand what needs to be proven}: simply $T(\vec{0}) = \vec{0}$ is insufficient in this case. One needs to prove linearity: $T(au + bv) = a T(u) + b T(v)$.
		
		\item \uline{Providing an example is not providing a proof}: this should be self-explanatory. An example can be used to illuminate why a statement is true, but it does not present any sort of rigour.
		
		\item \uline{Justify the steps using linearity}: if you did not write ``by properties of vector addition'', or ``by linearity, ...'', I overlooked it mostly. It is important to know \textit{how} steps go from one to another.
		
		
		\item \uline{Vector notation}: it is either
		\begin{align*}
			\begin{bmatrix}
			a \\ b \\ c
			\end{bmatrix} \quad \text{or} \quad (a,b,c)
		\end{align*}
		
	\end{itemize}
	
	
	
	\item[1.9.23] \begin{itemize}
		\item \uline{Justifications}: parts (b) and (c)---the proofs are completely not necessary. Simply stating that the rotation map corresponds to some linear map (an explicit formula can be found, as above), and that the composition of linear transformations is linear, will suffice.
	\end{itemize}
	
	
	\item[Extra Credit] For those who attempted the extra credit question:
	\begin{itemize}
		\item \uline{Know what you need to prove}: you want to prove that the solution sets are equivalent, so you need to prove subset relation holds both ways. In doing that, start with your assumptions, then work towards the proposition.
		
		\item \uline{Use all the assumptions in your proof}: for instance, that $k \in \R \setminus \set{0}$ guarantees that $k$ has a multiplicative inverse, that $\frac{1}{k}$ makes sense to write down. Not using that assumption is problematic.
		
		\item \uline{Generosity}: I know most of you have never written (or attempted to write) a formal proof before. I was lenient enough such that if you tried this problem, you would have gotten some mark(s) in return. In the future, \textit{think through what is being asked, and answer the question that needs to be addressed}.
	\end{itemize}

\end{enumerate}


%\clearpage
%
%Comments about assignment 1 and logistics:
%\begin{itemize}
%	\item \textbf{If a student wants to have part(s) of assignment 1 regraded/rechecked}, have him/her hand in the assignment next Monday. Have him/her circle the question of interest, let them write a justification for regrading, and I'll see to it.
%	
%	
%	\item \textbf{The grade distribution is as follows}:
%	
%	\begin{itemize}
%		\item Min: 33.00 
%		\item 1st Quartile: 70.50
%		\item Median: 83.00
%		\item Mean: 78.57
%		\item 3rd Quartile: 91.00
%		\item Max: 107.00
%		\item NA: 5
%	\end{itemize}
%
%	\item \textbf{General comments}:
%	\begin{itemize}
%		\item Know the definitions of concepts that are being asked.
%		\item 
%	\end{itemize}
%\end{itemize}
%
%


\end{document}