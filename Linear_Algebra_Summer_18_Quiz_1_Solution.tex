\documentclass[answers,11pt]{exam}
\usepackage[utf8]{inputenc}
\usepackage[letterpaper,top=1.4in,bottom=1.2in,right=1.3in,left=1.3in]{geometry}


\newcommand{\class}{Linear Algebra}
\newcommand{\term}{Summer 2018}
\newcommand{\examnum}{Quiz 1}
\newcommand{\examdate}{07.09.18}
\newcommand{\timelimit}{20 Minutes}

\pagestyle{head}
\firstpageheader{}{}{}
\runningheader{\class}{\examnum\ - Page \thepage\ of \numpages}{\examdate}
\runningheadrule

\usepackage{hyphenat}
\usepackage{graphicx}
\usepackage{tikz}
\usepackage{pgfplots}
\usepackage{mathrsfs}



\usepackage{array, booktabs, caption}
\usepackage{ragged2e}
\usepackage{multirow}
\usepackage{hhline}% http://ctan.org/pkg/hhline
\usepackage{makecell} 
\usepackage{enumitem}
\usepackage{bbm}
\usepackage{soul}

%\usepackage{mtpro2}


%\usepackage[nottoc]{tocbibind}
\usepackage[bottom]{footmisc}
\usepackage{natbib}
\usepackage{hyperref}
\usepackage{url}

\usepackage[normalem]{ulem}
\usepackage{relsize}
\usepackage{commath}
\usepackage{mathtools}
\allowdisplaybreaks


\usepackage{float}

\usepackage{color}
\usepackage{commath}
\usepackage{amsthm}

\newcommand\independent{\protect\mathpalette{\protect\independenT}{\perp}}
\def\independenT#1#2{\mathrel{\rlap{$#1#2$}\mkern2mu{#1#2}}}

\usepackage[utf8]{inputenc}
\usepackage[english]{babel}
\usepackage{textcomp}

\renewcommand\theadalign{lc}
\renewcommand\theadfont{\bfseries}
\usepackage{amssymb} %maths
\usepackage{amsmath} %maths

\usepackage[utf8]{inputenc} %useful to type directly diacritic characters

\usepackage{nomencl}
\makenomenclature

\usepackage{accents}

\makeatletter
\renewcommand*\env@matrix[1][*\c@MaxMatrixCols c]{%
	\hskip -\arraycolsep
	\let\@ifnextchar\new@ifnextchar
	\array{#1}}
\makeatother


\newtheoremstyle{mytheoremstyle} % name
{\topsep}                    % Space above
{\topsep}                    % Space below
{}                   % Body font
{}                           % Indent amount
{\bfseries}                   % Theorem head font
{.}                          % Punctuation after theorem head
{.5em}                       % Space after theorem head
{}  % Theorem head spec (can be left empty, meaning ‘normal’)


%\theoremstyle{mytheoremstyle}
%
\theoremstyle{definition}
\newtheorem{definition}{Definition}%[section]
\newtheorem{proposition}[definition]{Proposition}
\newtheorem{theorem}[definition]{Theorem}
\newtheorem{corollary}[definition]{Corollary}
\newtheorem{lemma}[definition]{Lemma}
\newtheorem{obs}[definition]{Observation}
\newtheorem{example}[definition]{Example}
\newtheorem{assumption}[definition]{Assumption}
\newtheorem{properties}[definition]{Properties}
\newtheorem{motivation}[definition]{Motivation}
\newtheorem{derivation}[definition]{Derivation}
\newtheorem{remark}[definition]{Remark}
\newtheorem{fact}[definition]{Fact}

\theoremstyle{definition}


\usepackage{caption}
\captionsetup[figure]{labelfont=sc}
\setlist[enumerate]{font=\bfseries\sffamily}



%\addto\captionsenglish{\renewcommand*{\proofname}{\scshape Proof.}}


%\numberwithin{equation}{section}

\DeclareMathOperator{\N}{\mathbb{N}}
\DeclareMathOperator{\Q}{\mathbb{Q}}
\DeclareMathOperator{\R}{\mathbb{R}}
\DeclareMathOperator{\Z}{\mathbb{Z}}
\DeclareMathOperator{\NN}{\mathcal{N}}
\DeclareMathOperator{\bindist}{\mathsf{B}}
\DeclareMathOperator{\DD}{D}
\DeclareMathOperator{\betadist}{Beta}
\DeclareMathOperator{\E}{\mathbf{E}}
\DeclareMathOperator{\cov}{Cov}
\DeclareMathOperator{\var}{Var}
\DeclareMathOperator{\samplespace}{\mathcal{S}}
\DeclareMathOperator{\suchthat}{\text{ s.t. }}
\DeclareMathOperator{\summod}{\accentset{\circ}{+}}
\DeclareMathOperator{\im}{im}
\DeclareMathOperator{\1}{\mathbbm{1}}
\DeclareMathOperator{\LL}{\mathscr{L}}
\DeclareMathOperator{\Imaginary}{Im}
\DeclareMathOperator{\supp}{Supp}
\DeclareMathOperator{\powerset}{\mathcal{P}}
\DeclareMathOperator{\normP}{norm}
\DeclareMathOperator{\BB}{\mathscr{B}}
\DeclareMathOperator{\contf}{\mathcal{C}}
\DeclareMathOperator{\riemannint}{\mathscr{R}}
\DeclareMathOperator{\osc}{osc}
\DeclareMathOperator{\sigmaalg}{\sigma-algebra}
\DeclareMathOperator{\MM}{\mathscr{M}}
\DeclareMathOperator{\diam}{diam}
\DeclareMathOperator{\D}{\dif}
\DeclareMathOperator{\Span}{span}
\DeclareMathOperator{\PP}{\mathbf{P}}
\DeclareMathOperator{\CC}{\mathscr{C}}
\DeclareMathOperator{\sgn}{sgn}

\renewcommand{\leq}{\leqslant}
\renewcommand{\geq}{\geqslant}
\renewcommand{\epsilon}{\varepsilon}
\renewcommand{\phi}{\varphi}

\newcommand{\Tau}{\mathcal{T}}

%\newcommand{\vect}[1][2]{\LL(#1,#2)}
\newcommand{\Lspace}[4]{\mathscr{L}^{#1}(#2,#3,#4)}
\newcommand{\condset}[4]{\left\{ #1  : \: #2 #3 #4 \right\}}
\newcommand{\ball}[2]{B(#1,#2)}
\newcommand{\innerproduct}[2]{\left\langle #1,#2 \right\rangle}


%\titleformat{\section}
%{\centering\Large\normalfont\scshape}{\thesection .}{0.5em}{}
%
%\titleformat{\subsection}
%{\centering\large\normalfont\scshape}{\thesubsection .}{0.5em}{}

\renewcommand{\qedsymbol}{$\blacksquare$}



\begin{document}

\noindent
\begin{tabular*}{\textwidth}{l @{\extracolsep{\fill}} r @{\extracolsep{6pt}} l}
\textbf{\class} & \textbf{Name:} & \makebox[2in]{\hrulefill}\\
\textbf{\term} &&\\
\textbf{\examnum} &&\\
\textbf{\examdate} &&\\
\textbf{Time Limit: \timelimit} 
\end{tabular*}\\
\rule[2ex]{\textwidth}{2pt}

\noindent This quiz contains 2 sides (including this cover page) and 4 questions. Total of points is \numpoints.


\begin{center}
Grade Table (for grader use only)\\
\addpoints
\gradetable[v][questions]
\end{center}

\noindent
\rule[2ex]{\textwidth}{2pt}

\noindent Few things to note:
\begin{itemize}
	\item Unless otherwise stated, the objects stated here are vectors, and $A$ denotes a matrix of a specified dimension.
	\item Marks breakdown are indicated \textit{after} the solution.
\end{itemize}

\bigskip

\begin{questions}

\question[25] State the definition of:
\noaddpoints
\begin{parts}
\part[12.5]The span of the vectors $\vec{v}_1,\ldots, \vec{v}_m$ in $\mathbb{R}^n$.

\begin{solution}
	The (linear) \textbf{span} of vectors $\set{v_1,v_2,\dots,v_m} \in \R^n$, denoted
	$$\Span \set{v_1,v_2.\dots,v_m}$$
	is the set of all linear combinations of the collection of vectors. This is a vector space (which will be proved later on in this course).
	
	\textit{[5 for equivalent statement of ``all linear combinations of vectors $\set{v_1,v_2,\dots,v_m}  \\ \in \R^n$'', 5 for the equivalent statement of ``set''; 2.5 for having both in the response]}
\end{solution}


\part[12.5] Linear independence of $\{\vec{v}_1,\ldots, \vec{v}_m\}$ in $\mathbb{R}^n$.

\begin{solution}
	A collection of vectors is said to be \textbf{linearly independent} if the equation
	\begin{align}
	\label{eq1}
	\tag{$\star$}
	\sum_{k=1}^{m} a_k v_k = a_1 v_1 + a_2 v_2 + \dots + a_m v_m = 0
	\end{align}
	admits only the trivial solution, ie. $a_1 = \dots = a_m = 0$.
	
	\textit{[5 for writing \eqref{eq1}, 5 for equivalent statement of ``trivial solution'', 2.5 for having both in the response. Other equivalent responses were accepted]}
\end{solution}

\end{parts}
\addpoints

\question[25] Determine whether or not the vector $\vec{b}$ is a linear combination of vectors $\vec{a}_1,\vec{a}_2,\vec{a}_3$ where
\begin{equation*}
\vec{b}=
\begin{bmatrix}
11\\
-5\\
9
\end{bmatrix}, 
\vec{a}_1=
\begin{bmatrix}
1\\
0\\
1
\end{bmatrix},
\vec{a}_2=
\begin{bmatrix}
-2\\
3\\
-2
\end{bmatrix},
\vec{a}_3=
\begin{bmatrix}
-6\\
7\\
5
\end{bmatrix}.
\end{equation*}
\addpoints

\begin{solution}
	
	\uline{$b$ is a linear combination of vectors $\set{a_1,a_2,a_3}$}. To see this, write the matrix in augmented form and row reduce, until we notice that the system admits a unique solution.
	
	\begin{align*}
	\begin{bmatrix}[c c c | c]
	1 & -2 & -6 & 11 \\ & 3 & 7 & -5 \\ 1 & -2 & 5 & 9
	\end{bmatrix} & \leadsto \begin{bmatrix}[c c c | c]
	1 & -2 & -6 & 11 \\ & 3 & 7 & -5 \\  &  & 11 & -2
	\end{bmatrix} \\
	&\leadsto \begin{bmatrix}[c c c | c]
	1 & -2 & -6 & 11 \\ & 3 & 7 & -5 \\  &  & 1 & -\frac{2}{11}
	\end{bmatrix} \\
	\end{align*}
	Reading off the coefficients give us the following solutions:
	\begin{align*}
	\begin{cases}
	a_1 &= 11 - \frac{82}{33} - \frac{12}{11} =  \frac{245}{33} \\
	a_2 &= \frac{-5 + \frac{14}{11}}{3} = -\frac{41}{33} \\
	a_3 &= -\frac{2}{11}
	\end{cases}
	\end{align*}
	or, equivalently (up to rescaling),
	\begin{align*}
	\begin{cases}
	a_1 &= \frac{173}{33} \\
	a_2 &= \frac{85}{3} \\
	a_3 &= -\frac{20}{11}
	\end{cases}
	\end{align*}
	
	
	\textit{[5 for writing the matrix in augmented form, 5 for solving up to upper triangular form \textit{correctly}, 5 for correct conclusion, 5 for correct justification, 5 for having all of the above]}
\end{solution}

\question[25] Descibe the solutions to the equation $A\vec{x}=\vec{0}$ in parametric form, where the matrix $A$ is row equivalent to 
\begin{equation*}
\begin{bmatrix}
1&3&0&-4\\
2&6&0&-8
\end{bmatrix}.
\end{equation*}
That is to say: express the basic variable(s) in terms of the free variable(s). (Hint: The matrix above is not yet augmented, remember to augment it with $\vec{0}$.)
\addpoints


\begin{solution}
	
	Per the hint given in the question, we attempt to solve the following augmented system (and we use the convention that the variable $x_i$ is assigned to the $i$-th column):
	\begin{align*}
	\begin{bmatrix}[c c c c | c]
	1 & 3 & 0 & -4 & 0 \\ 2 & 6 & 0 & -8 & 0
	\end{bmatrix} \leadsto \begin{bmatrix}[c c c c | c]
	1 & 3 & 0 & -4 & 0 \\ 0 & 0 & 0 & 0 & 0
	\end{bmatrix}
	\end{align*}
	so the system has 1 pivot column (the first one), and the rest are free variables. Also, the solution set does depend on $x_3$ by rank-nullity theorem. Hence, we have the solution set
	\begin{align}
	\label{eq2}
	\tag{$\ast$}
	S = \condset{ x_2 \begin{pmatrix} -3 \\ 1 \\ 0 \\ 0 \\\end{pmatrix} + x_3 \begin{pmatrix} 0 \\ 0 \\ 1 \\ 0 \end{pmatrix} + x_4 \begin{pmatrix} 4 \\ 0 \\ 0 \\ 1 \end{pmatrix} }{x_2, x_3, x_4}{ \in}{ \R}
	\end{align}
	
	
	\textit{[5 for writing the matrix in augmented form, 5 for \textit{correctly} row reducing the matrix, 5 for \textit{correctly} stating the free variables and pivot column, 5 for a correct set (as in \eqref{eq2})---or an equivalent statement, 5 for having all of the above]}
\end{solution}




\question[25] Find the standard matrix representation for the linear transformation $T:\mathbb{R}^2\to \mathbb{R}^2$ which sends $\vec{e}_1$ to itself and $\vec{e}_2$ to $3\vec{e}_1+7\vec{e}_2$. That is, $T(\vec{e}_1)=\vec{e}_1$ and $T(\vec{e}_2)=3\vec{e}_1+7\vec{e}_2$. 

\begin{solution}
	
	With the image of the canonical basis under $T$ already given, we know we just need to find $T(e_1)$ and $T(e_2)$, and the corresponding standard matrix of linear transformation $M$ is given by
	\begin{align*}
	M = \begin{bmatrix}
	| & | \\ T(e_1) & T(e_2) \\ | & |
	\end{bmatrix}
	\end{align*}
	so we calculate and see that
	\begin{align}
	\nonumber T(e_1) &= T\left( \begin{bmatrix} 1 \\ 0 \end{bmatrix} \right) \\
	\label{eq3} \tag{$\star \star$} &= \begin{bmatrix} 1 \\ 0 \end{bmatrix} \\
	\nonumber T(e_2) &= T\left( \begin{bmatrix} 0 \\ 1 \end{bmatrix} \right) \\
	\nonumber &= 3 \begin{bmatrix} 1 \\ 0 \end{bmatrix} + 7 \begin{bmatrix} 0 \\ 1 \end{bmatrix} \\
	\label{eq4} \tag{$\ast \ast$} &= \begin{bmatrix} 3 \\ 7 \end{bmatrix}
	\end{align}
	so we have the matrix of linear transformation
	\begin{align}
	\label{eq5}
	\tag{$\dagger$}
	M = \begin{bmatrix}[c c]
	1 & 3 \\ 0 & 7
	\end{bmatrix}
	\end{align}
	
	
	\textit{[5 for stating/acknowledging the definition of the standard matrix, 5 for $T(e_1)$ calculation (as in \eqref{eq3}), 5 for $T(e_2)$ calculation (as in \eqref{eq4}), 5 for the correct matrix $M$ (as in \eqref{eq5}), 5 for all of the above]}
\end{solution}

\end{questions}


\clearpage


\section*{General Comments}
The following are general observations from the responses collected.


\begin{enumerate}
	\item \begin{itemize}
		
		\item \uline{Know what objects you are dealing with}. It is important to note, in a definition, what the concept is as a mathematical object. A linear transformation, for example, is a \textit{map} from some space to another. The span of a collection of vectors is a \textit{set}. Knowing what objects you are dealing with will go a long way in your career in mathematics. This has caused a harsh grading at first (which prompted a regrading because we are still early on in the course).
		
		\item \uline{Avoid passing consequences of a concept as a definition}. Granted, there are many grey areas in these definitions; some are more acceptable than others. Should this arise in the future, the course instructor will have the final say-so. We do accept, however, \textit{equivalent} definitions---hence the qualifier ``equivalent statement(s)'' in the markscheme.
	\end{itemize}
	
	
	\item \uline{Don't guess and check}. Some responses involved guessing and checking some combinations of $(\vec{a}_i)_{i=1}^{3}$, but, as the solution set tells you, the particular solution is nontrivial. One needs to work through the problem by thinking: \textit{what is it that the question is asking?} We want to verify if $\vec{b}$ is a linear combination of $(\vec{a}_i)_{i=1}^{3}$; we know that if $A\vec{x} = \vec{b}$ admits a solution, then $\vec{b} \in C(A)$ (column space of $A$; ie. $\vec{b} = \sum_{k=1}^{3} c_i \vec{a}_i$ for some collection of $c_i \in \R$, $i=1,2,3$). So the question boils down to checking if $A\vec{x} = \vec{b}$ is consistent or not.
	
	
	\item \begin{itemize}
		\item \uline{Check the dimensions of the vectors in the solution set}. This will be easier to understand once we know what a vector space is, and its subspaces.
		
		\item \uline{Check row reductions carefully}.
	\end{itemize}


	\item \begin{itemize}
		\item \uline{Linear transformation $T$ is a map between two vector spaces that satisfies linearity and scalar multiplicative properties---which admits a matrix representation}. It is incorrect to say ``$T = $ matrix''; rather, one would say a matrix $M$ represents the linear transformation $T$.
		
		
		\item \uline{Know the definition of a concept}. A standard matrix is one whose columns contain the image of the canonical basis under $T$; just knowing that will get you some marks.
		
		
		\item \uline{Check your calculations carefully}.
	\end{itemize}
\end{enumerate}



\end{document}
