\documentclass{amsart} %leqno
\usepackage[left=1.3in,right=1.3in,bottom=1.6in]{geometry}
%\usepackage{titlesec}
%\usepackage{lipsum}
\usepackage{mathrsfs}
%\usepackage{libertine}
%\usepackage[lite,subscriptcorrection,nofontinfo,amsbb,eucal]{mtpro2}

%\usepackage{lmodern}

\linespread{1}

%\usepackage{sansmathfonts}
%\usepackage{cmbright}
%\usepackage[T1]{fontenc}
%\renewcommand*\familydefault{\sfdefault}

%slantedGreek

\usepackage{hyphenat}
\usepackage{graphicx}
\usepackage{tikz}
\usepackage{pgfplots}
\usepackage{accents}


\usepackage{array, booktabs, caption}
\usepackage{ragged2e}
\usepackage{multirow}
\usepackage{hhline}% http://ctan.org/pkg/hhline
\usepackage{makecell} 
\usepackage{enumitem}
\usepackage{bbm}
\usepackage{soul}

%\usepackage{mtpro2}


%\usepackage[nottoc]{tocbibind}
\usepackage[bottom]{footmisc}
%\usepackage{natbib}


\usepackage[normalem]{ulem}
\usepackage{relsize}
\usepackage{commath}
\usepackage{mathtools}
\allowdisplaybreaks



%\usepackage{fancyhdr}
%
%\pagestyle{fancy}
%\fancyhf{}
%%\renewcommand{\headrulewidth}{0.5pt}
%%\rhead{\textsc{}}
%%\cfoot{\textsc{Honors Analysis II}}
%%\lhead{\textsc{Meng Hsuan (Rex) Hsieh}}
%%\cfoot{}
%\fancyhead[C]{\rightmark}
%\fancyhead[RO,LE]{\thepage}

\usepackage{float}

\usepackage{color}
\usepackage{commath}
\usepackage{amsthm}

\newcommand\independent{\protect\mathpalette{\protect\independenT}{\perp}}
\def\independenT#1#2{\mathrel{\rlap{$#1#2$}\mkern2mu{#1#2}}}

\usepackage[utf8]{inputenc}
\usepackage[english]{babel}
\usepackage{textcomp}

\renewcommand\theadalign{lc}
\renewcommand\theadfont{\bfseries}
\usepackage{amssymb} %maths
\usepackage{amsmath} %maths

\usepackage{hyperref}
\hypersetup{colorlinks=true,
	linkcolor=blue,          % color of internal links (change box color with linkbordercolor)
	citecolor=green,        % color of links to bibliography
	filecolor=magenta,      % color of file links
	urlcolor=cyan           % color of external links
}

\renewcommand{\epsilon}{\varepsilon}


\newtheoremstyle{mytheoremstyle} % name
{\topsep}                    % Space above
{\topsep}                    % Space below
{}                   % Body font
{}                           % Indent amount
{\bfseries}                   % Theorem head font
{.}                          % Punctuation after theorem head
{.5em}                       % Space after theorem head
{}  % Theorem head spec (can be left empty, meaning ‘normal’)


\theoremstyle{mytheoremstyle}

\theoremstyle{definition}
\newtheorem{definition}{Definition}[section]
\newtheorem*{rmk}{Remark}
\newtheorem{proposition}[definition]{Proposition}
\newtheorem{theorem}[definition]{Theorem}
\newtheorem{corollary}[definition]{Corollary}
\newtheorem{lemma}[definition]{Lemma}
\newtheorem{obs}[definition]{Observation}
\newtheorem{example}[definition]{Example}
\newtheorem{assumption}[definition]{Assumption}
\newtheorem{properties}[definition]{Properties}
\newtheorem{motivation}[definition]{Motivation}
\newtheorem{assertion}[definition]{Assertion}
\newtheorem{derivation}[definition]{Derivation}
\newtheorem{remark}[definition]{Remark}
\newtheorem{fact}[definition]{Fact}
\newtheorem{consequence}[definition]{Consequence}

%\addto\captionsenglish{\renewcommand*{\proofname}{\scshape Proof.}}


%\numberwithin{equation}{section}

\DeclareMathOperator{\N}{\mathbb{N}}
\DeclareMathOperator{\Q}{\mathbb{Q}}
\DeclareMathOperator{\R}{\mathbb{R}}
\DeclareMathOperator{\Z}{\mathbb{Z}}
\DeclareMathOperator{\Com}{\mathbb{C}}
\DeclareMathOperator{\NN}{\mathcal{N}}
\DeclareMathOperator{\bindist}{\mathsf{B}}
\DeclareMathOperator{\DD}{D}
\DeclareMathOperator{\betadist}{Beta}
\DeclareMathOperator{\E}{\mathbf{E}}
\DeclareMathOperator{\cov}{Cov}
\DeclareMathOperator{\var}{Var}
\DeclareMathOperator{\samplespace}{\mathcal{S}}
\DeclareMathOperator{\suchthat}{\text{ s.t. }}
\DeclareMathOperator{\summod}{\accentset{\circ}{+}}
\DeclareMathOperator{\im}{im}
\DeclareMathOperator{\1}{\mathbbm{1}}
\DeclareMathOperator{\LL}{\mathscr{L}}
\DeclareMathOperator{\Imaginary}{Im}
\DeclareMathOperator{\supp}{Supp}
\DeclareMathOperator{\powerset}{\mathcal{P}}
\DeclareMathOperator{\normP}{norm}
\DeclareMathOperator{\BB}{\mathscr{B}}
\DeclareMathOperator{\contf}{\mathcal{C}}
\DeclareMathOperator{\riemannint}{\mathscr{R}}
\DeclareMathOperator{\osc}{osc}
\DeclareMathOperator{\sigmaalg}{\sigma-algebra}
\DeclareMathOperator{\MM}{\mathcal{M}}
\DeclareMathOperator{\diam}{diam}
\DeclareMathOperator{\D}{\dif}
\DeclareMathOperator{\Span}{span}
\DeclareMathOperator{\PP}{\mathbf{P}}
\DeclareMathOperator{\B}{\mathcal{B}}
\DeclareMathOperator{\CC}{\mathcal{C}}
\DeclareMathOperator{\sgn}{sgn}

\DeclareMathOperator{\col}{\mathsf{C}}
\DeclareMathOperator{\nul}{\mathsf{N}}

\renewcommand{\leq}{\leqslant}
\renewcommand{\geq}{\geqslant}
\renewcommand{\epsilon}{\varepsilon}
\renewcommand{\phi}{\varphi}
\newcommand{\zerov}{\mathbf{0}}
\newcommand{\Tau}{\mathcal{T}}
\newcommand{\rng}{\mathsf{R}}

%\newcommand{\vect}[1][2]{\LL(#1,#2)}
\newcommand{\Lspace}[4]{\mathscr{L}^{#1}(#2,#3,#4)}
\newcommand{\condset}[4]{\left\{ #1  : \: #2 #3 #4 \right\}}
\newcommand{\ball}[2]{B(#1,#2)}
\newcommand{\innerproduct}[2]{\left\langle #1,#2 \right\rangle}
\newcommand{\polyn}[2]{\mathcal{P}_{#1}(#2)}
\newcommand{\GL}[2]{\mathrm{GL}_{#1}(#2)}
\newcommand{\proj}[2]{\mathrm{Proj}_{#1} {#2} }
\newcommand{\SL}[2]{\mathrm{SL}_{#1}(#2)}

%\titleformat{\section}
%{\centering\Large\normalfont\scshape}{\thesection .}{0.5em}{}
%
%\titleformat{\subsection}
%{\centering\large\normalfont\scshape}{\thesubsection .}{0.5em}{}

\renewcommand{\qedsymbol}{$\blacksquare$}

\begin{document}

\title{Linear Algebra: (Fairly Difficult) Practice Final Exam}

\maketitle

\noindent These are some difficult (but doable) practice problems for the linear algebra final exam. These are presented in an increasing level of difficulty. \\\\


\begin{enumerate}[itemsep=1em]

	\item Prove that similar matrices have the same characteristic polynomial.

	\item Let 
	\begin{align*}
		W = \begin{bmatrix}
		0 & -2 \\ 1 & 3
		\end{bmatrix}
	\end{align*}
	Show that $W$ is diagonalizable, by finding a $Q \in \MM_2(\R)$ such that $Q^{-1} W Q $ is a diagonal matrix. Derive a formula for $W^{n}$ for some arbitrary $n \in \N$.


	\item Prove or disprove (ie. true or false):
	\begin{enumerate}
		\item Any linear transformation on a $n$-dimensional vector space that has less than $n$ distinct eigenvalues is not diagonalizable.
		
		\item Two distinct eigenvectors corresponding to the same eigenvalue are always linearly dependent.
		
		\item If $\lambda$ and $\eta$ are distinct eigenvalues of a linear transformation $T$, then $E_{\lambda} \cap E_{\eta} = \set{0}$, where $E_{\lambda}$ is the eigenspace corresponding to the eigenvalue $\lambda$.
		
		\item Let $A \in \MM_{n} (\R)$. If $A$ is diagonalizable, then $A^{-1}$ is diagonalizable.
		
		\item The $m \times n$ zero matrix is the only $m \times n$ matrix of rank $0$.
	\end{enumerate}

	\item Prove or disprove: let $A \in \MM_{n}(\R)$.
	\begin{enumerate}
		\item If $A$ is diagonalizable, then $A^2$ is diagonalizable.
		\item If $A^2$ is diagonalizable, then $A$ is diagonalizable.
	\end{enumerate}

	\item Let $V = \polyn{3}{\R}$. Let $T(a+bx+cx^2+dx^3) = -d + (-c+d)x + (a+b-2c)x^2 + (-b+c-2d)x^3$, and $\mathcal{B} = \set{1-x+x^3, 1+x^2, 1, x+x^2}$.

	Show that $T$ is a linear transformation. Then, find the matrix of linear transformation with respect to the basis $\mathcal{B}$ described above.
	
	
	\item Let $V = \MM_2(\R)$. Let
	\begin{align*}
	T \left( \begin{bmatrix}
	a & b \\ c & d
	\end{bmatrix} \right) = \begin{bmatrix}
	-7a - 4b + 4c - 4d & b \\ -8a -4b + 5c -4d & d
	\end{bmatrix}
	\end{align*}
	and
	\begin{align*}
	\mathcal{C} = \set{ \begin{bmatrix}
		1 & 0 \\ 1 & 0
		\end{bmatrix}, \begin{bmatrix}
		-1 & 2 \\ 0 & 0
		\end{bmatrix}, \begin{bmatrix}
		1 & 0 \\ 2 & 0
		\end{bmatrix}, \begin{bmatrix}
		-1 & 0 \\ 0 & 2
		\end{bmatrix} }
	\end{align*}
	
	Show that $T$ is a linear transformation. Then, find the matrix of linear transformation with respect to the basis $\mathcal{C}$ described above.


	\item A \textit{scalar matrix} is a square matrix of the form $\lambda I$ for some scalar $\lambda$.
	
	\begin{enumerate}
		\item Show that if a square matrix $A$ is similar to a scalar matrix $\lambda I$, then $A = \lambda I$.
		\item Show that a diagonalizable matrix having only one eigenvalue is a scalar matrix.
		\item Show that
		\begin{align*}
		\begin{bmatrix}
		1 & 1 \\ & 1
		\end{bmatrix}
		\end{align*}
		is not diagonalizable.
	\end{enumerate}

	\item Let $A \in \MM_{m \times n}$, and suppose $v$ and $w$ are orthogonal eigenvectors of $A^{\intercal}A$. Show that $Av$ and $Aw$ are orthogonal.
	
	\item Let $V$ be an inner product space, and $S = \set{x_i}_{i=1}^{n}$ be an orthonormal subset in $V$. Fix $x \in V$. Then the following holds:
	\begin{equation}
		\label{eq1}
		\tag{$\star$}
		\sum_{k=1}^{n} \innerproduct{x}{x_k} \leq \norm{x}^2
	\end{equation}
	This is referred to as \textit{Bessel's inequality}. For the purpose of the problem, assume the result has already been proven. As a reminder, $\norm{x} = \sqrt{\innerproduct{x}{x}}$.
	
	Show that Bessel's inequality \eqref{eq1} is an equality if and only if $x \in \Span(S)$.
	
	\begin{rmk}
		Note that two consequences follow from \eqref{eq1}: the series $\sum_{k=1}^{\infty} \innerproduct{x}{x_k}$ converges by taking limits of both sides, and just the fact that we can find an orthonormal set in $V$ can allow us to find the norm by an less cumbersome calculation.
	\end{rmk}

	
	\item Let $A \in \MM_{n}(\R)$. $A$ is \textit{nilpotent} if $A^k = 0$ (where $0$ is the zero matrix) for some $k \in \N$. For instance, $\begin{bmatrix}
	0 & 1 \\ 0 & 0 
	\end{bmatrix}$ is nilpotent.
	\begin{enumerate}
		\item Let $\lambda$ be an eigenvalue to a nilpotent matrix $A$. Show that $\lambda = 0$. Hint: proceed by definition.
		
		\item Show that if $A$ is both nilpotent and diagonalizable, then $A$ is the zero matrix. Hint: use the previous part.
		
		\item Let $B$ be the matrix representation of the following linear transformation:
		\begin{align*}
		\fullfunction{S}{\polyn{5}{R}}{\polyn{5}{R}}{p(x)}{\dod{p}{x}}
		\end{align*}
		Without doing any calculations, explain why $B$ must be nilpotent.
	\end{enumerate}

\end{enumerate}


	
\end{document}