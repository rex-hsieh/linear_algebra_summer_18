\documentclass[answers,11pt]{exam}
\usepackage[utf8]{inputenc}
\usepackage[letterpaper,top=1.4in,bottom=1.2in,right=1.3in,left=1.3in]{geometry}

\usepackage{amsmath,amssymb}
%\usepackage{commath}
%\usepackage{mathtools}
%\usepackage{multicol}

\newcommand{\class}{Linear Algebra}
\newcommand{\term}{Summer 2018}
\newcommand{\examnum}{Quiz 4: Suggested Solutions}
\newcommand{\examdate}{07.30.18}
\newcommand{\timelimit}{20 Minutes}

\pagestyle{head}
\firstpageheader{}{}{}
\runningheader{\class}{\examnum\ - Page \thepage\ of \numpages}{\examdate}
\runningheadrule

\usepackage{hyphenat}
\usepackage{graphicx}
\usepackage{tikz}
\usepackage{pgfplots}
\usepackage{mathrsfs}

\usepackage{array, booktabs, caption}
\usepackage{ragged2e}
\usepackage{multirow}
\usepackage{hhline}% http://ctan.org/pkg/hhline
\usepackage{makecell} 
\usepackage{enumitem}
\usepackage{bbm}
\usepackage{soul}

%\usepackage{mtpro2}
%\usepackage[nottoc]{tocbibind}

\usepackage[bottom]{footmisc}
\usepackage{natbib}
\usepackage{hyperref}
\usepackage{url}

\usepackage[normalem]{ulem}
\usepackage{relsize}
\usepackage{commath}
\usepackage{mathtools}
\allowdisplaybreaks

\usepackage{float}

\usepackage{color}
\usepackage{commath}
\usepackage{amsthm}

\newcommand\independent{\protect\mathpalette{\protect\independenT}{\perp}}
\def\independenT#1#2{\mathrel{\rlap{$#1#2$}\mkern2mu{#1#2}}}

\usepackage[utf8]{inputenc}
\usepackage[english]{babel}
\usepackage{textcomp}

\renewcommand\theadalign{lc}
\renewcommand\theadfont{\bfseries}
\usepackage{amssymb} %maths
\usepackage{amsmath} %maths

\usepackage[utf8]{inputenc} %useful to type directly diacritic characters

\usepackage{nomencl}
\makenomenclature

\usepackage{accents}

\makeatletter
\renewcommand*\env@matrix[1][*\c@MaxMatrixCols c]{%
	\hskip -\arraycolsep
	\let\@ifnextchar\new@ifnextchar
	\array{#1}}
\makeatother

\makeatletter
\renewcommand*\env@matrix[1][\arraystretch]{%
	\edef\arraystretch{#1}%
	\hskip -\arraycolsep
	\let\@ifnextchar\new@ifnextchar
	\array{*\c@MaxMatrixCols c}}
\makeatother


\newtheoremstyle{mytheoremstyle} % name
{\topsep}                    % Space above
{\topsep}                    % Space below
{}                   % Body font
{}                           % Indent amount
{\bfseries}                   % Theorem head font
{.}                          % Punctuation after theorem head
{.5em}                       % Space after theorem head
{}  % Theorem head spec (can be left empty, meaning ‘normal’)


\theoremstyle{definition}
\newtheorem{definition}{Definition} %[section]
\newtheorem{proposition}[definition]{Proposition}
\newtheorem{theorem}[definition]{Theorem}
\newtheorem{corollary}[definition]{Corollary}
\newtheorem{lemma}[definition]{Lemma}
\newtheorem{obs}[definition]{Observation}
\newtheorem{example}[definition]{Example}
\newtheorem{assumption}[definition]{Assumption}
\newtheorem{properties}[definition]{Properties}
\newtheorem{motivation}[definition]{Motivation}
\newtheorem{derivation}[definition]{Derivation}
\newtheorem{remark}[definition]{Remark}
\newtheorem{fact}[definition]{Fact}

\usepackage{caption}
\captionsetup[figure]{labelfont=sc}
\setlist[enumerate]{font=\bfseries\sffamily}



%\addto\captionsenglish{\renewcommand*{\proofname}{\scshape Proof.}}


%\numberwithin{equation}{section}

\DeclareMathOperator{\N}{\mathbb{N}}
\DeclareMathOperator{\Q}{\mathbb{Q}}
\DeclareMathOperator{\R}{\mathbb{R}}
\DeclareMathOperator{\Z}{\mathbb{Z}}
\DeclareMathOperator{\NN}{\mathcal{N}}
\DeclareMathOperator{\bindist}{\mathsf{B}}
\DeclareMathOperator{\DD}{D}
\DeclareMathOperator{\betadist}{Beta}
\DeclareMathOperator{\E}{\mathbf{E}}
\DeclareMathOperator{\cov}{Cov}
\DeclareMathOperator{\var}{Var}
\DeclareMathOperator{\samplespace}{\mathcal{S}}
\DeclareMathOperator{\suchthat}{\text{ s.t. }}
\DeclareMathOperator{\summod}{\accentset{\circ}{+}}
\DeclareMathOperator{\im}{im}
\DeclareMathOperator{\1}{\mathbbm{1}}
\DeclareMathOperator{\LL}{\mathscr{L}}
\DeclareMathOperator{\Imaginary}{Im}
\DeclareMathOperator{\supp}{Supp}
\DeclareMathOperator{\powerset}{\mathcal{P}}
\DeclareMathOperator{\normP}{norm}
\DeclareMathOperator{\BB}{\mathcal{B}}
\DeclareMathOperator{\contf}{\mathcal{C}}
\DeclareMathOperator{\riemannint}{\mathscr{R}}
\DeclareMathOperator{\osc}{osc}
\DeclareMathOperator{\sigmaalg}{\sigma-algebra}
\DeclareMathOperator{\MM}{\mathscr{M}}
\DeclareMathOperator{\M}{\mathcal{M}}
\DeclareMathOperator{\diam}{diam}
\DeclareMathOperator{\D}{\dif}
\DeclareMathOperator{\Span}{span}
\DeclareMathOperator{\PP}{\mathbf{P}}
\DeclareMathOperator{\CC}{\mathbb{C}}
\DeclareMathOperator{\sgn}{sgn}

\DeclareMathOperator{\col}{Col}
\DeclareMathOperator{\nul}{Null}

\renewcommand{\leq}{\leqslant}
\renewcommand{\geq}{\geqslant}
\renewcommand{\epsilon}{\varepsilon}
\renewcommand{\phi}{\varphi}

\newcommand{\Tau}{\mathcal{T}}
\newcommand{\zerov}{\mathbf{0}}

\newcommand{\uu}{\mathbf{u}}
\newcommand{\vv}{\mathbf{v}}

%\newcommand{\vect}[1][2]{\LL(#1,#2)}
\newcommand{\Lspace}[4]{\mathscr{L}^{#1}(#2,#3,#4)}
\newcommand{\condset}[4]{\left\{ #1  : \: #2 #3 #4 \right\}}
\newcommand{\ball}[2]{B(#1,#2)}
\newcommand{\innerproduct}[2]{\left\langle #1,#2 \right\rangle}
\newcommand{\polyn}[2]{\mathcal{P}_{#1}(#2)}

%\titleformat{\section}
%{\centering\Large\normalfont\scshape}{\thesection .}{0.5em}{}
%
%\titleformat{\subsection}
%{\centering\large\normalfont\scshape}{\thesubsection .}{0.5em}{}

\renewcommand{\qedsymbol}{$\blacksquare$}


\begin{document}

\noindent
\begin{tabular*}{\textwidth}{l @{\extracolsep{\fill}} r @{\extracolsep{6pt}} l}
\textbf{\class} & \textbf{Name:} & \makebox[2in]{\hrulefill}\\
\textbf{\term} &&\\
\textbf{\examnum} &&\\
\textbf{\examdate} &&\\
\textbf{Time Limit: \timelimit} 
\end{tabular*}\\
\rule[2ex]{\textwidth}{2pt}

This quiz contains 2 sides (including this cover page) and 4 questions.\\
Total of points is \numpoints.


\begin{center}
Grade Table (for grader use only)\\
\addpoints
\gradetable[v][questions]
\end{center}

\noindent
\rule[2ex]{\textwidth}{2pt}

\begin{questions}

\question[30] Let $A$ denote an $n\times n$ matrix.
\noaddpoints
\begin{parts}
	\part[15] Define what it means for $v\in \mathbb{R}^n$ to be an eigenvector for $A$ with eigenvalue $\lambda\in \mathbb{R}$.
	
	\begin{solution}
		An \textit{eigenvector} of $A \in \M_{n \times n}$ is a nonzero vector $x$ such that $Ax = \lambda x$ for some scalar $\lambda \in \R$. To be clear, a scalar $\lambda$ is called an \textit{eigenvalue} of $A$ if there is a nontrivial solution $x$ to $Ax = \lambda x$; such a $x$ is called an \textit{eigenvector} corresponding to $\lambda$.
		
		\textit{[5 for any equivalent statement of $Ax = \lambda x$, 5 for stating ``nontrivial'' or ``nonzero'' vector $x$, 3 for defining eigenvector as a vector corresponding to some $\lambda$ (counting multiplicities), 2 for having all of the above]}
	\end{solution}
	
	\part[15] Define what it means for another $n\times n$ matrix $B$ to be similar to $A$.
	
	\begin{solution}
		Let $A, B \in \M_{n \times n}$. $B$ is \textit{similar} to $A$ if there exists an \textit{invertible matrix} $Q$ such that $B = QAQ^{-1}$ (or, up to necessary changes, $A = Q^{-1}BQ$). This is otherwise known as the \textit{conjugate} of $B$ to $A$.
		
		\textit{[6 for correctly stating either $B = QAQ^{-1}$ or the alternate form, 6 for ``there exists an invertible matrix'', 3 for having all of the above]}
	\end{solution}
	
\end{parts}
\addpoints




\question[20] Given that 
\begin{equation*}
P=\begin{bmatrix}
2&1\\
1&1
\end{bmatrix}
\end{equation*}
and 
\begin{equation*}
A=P \underbrace{\begin{bmatrix}
1&1\\
0&1
\end{bmatrix}}_{\coloneqq U} P^{-1}
\end{equation*}
compute $A^{10}$. 
\addpoints

\begin{solution}
	Notice that $P$ is invertible because $\det(P) = 21 - 20 = 1 \neq 0$, so the similar (conjugate) form above makes sense. Notice, also, that
	\begin{align*}
	A^{10} &= \underbrace{ (PUP^{-1}) (PUP^{-1}) \cdots (PUP^{-1})}_{10 \text{ times}} & \\
	&=\underbrace{ PU (P^{-1}P) U (P^{-1} P) \cdots (P^{-1}P) UP^{-1})}_{10 \text{ times}}  & \text{associativity} \\
	&= P U^{10} P^{-1} 
	\end{align*}
	Then, calculations yield that
%	\begin{align*}
%	P^{-1} &= \begin{bmatrix}
%	3 & -5 \\ -4 & 7
%	\end{bmatrix} \\
%	U^{10} &= \begin{bmatrix}
%	1 & 10 \\ & 1
%	\end{bmatrix}
%	\end{align*}
%	and combining these give
%	\begin{align*}
%	A^{10} = \begin{bmatrix}
%	-279 & 490 \\ -160 & 281
%	\end{bmatrix}
%	\end{align*}
	\begin{align*}
		P^{-1} &= \begin{bmatrix}
		1 & -1 \\ -1 & 2
		\end{bmatrix} \\
		U^{10} &= \begin{bmatrix}
		1 & 10 \\ & 1
		\end{bmatrix}
	\end{align*}
	and combining these give
	\begin{align*}
	A^{10} &= \begin{bmatrix}
	-19 & 40 \\ -10 & 21
	\end{bmatrix}
	\end{align*}
	
	\textit{[2 for attempt of any kind, 3 for $A^{10} = P U^{10} P^{-1} $, 3 for correct $P^{-1}$, 3 for correct $U^{10}$, 3 for attempt at calculating $A^{10}$ by brute force or any method, 3 for correct $A^{10}$, 3 for having all of the above]}
\end{solution}


\question[25] Compute the characteristic polynomial of 
\begin{equation*}
A=\begin{bmatrix}
1&0&0\\
0&1&1\\
0&1&1
\end{bmatrix}
\end{equation*}
and list the eigenvalues. 
\addpoints

\begin{solution}
	The characteristic polynomial is given by $\det(A - \lambda I) = 0$, where $I$ is the $3 \times 3$ identity matrix. Then, we have
	\begin{align*}
	\det(A - \lambda I) &= (1-\lambda)^3 - (1-\lambda) \\
	&= (1-\lambda)  ( (1-\lambda)^2 - 1 ) \\
	&= (1-\lambda)(-\lambda)(2-\lambda) = 0
	\end{align*}
	so the eigenvalues are $1,0,2$.
	
	\textit{[5 for attempt of any kind, 5 for correct setup of $\det(A-\lambda I) = 0$, 5 for correctly calculating the characteristic polynomial, 5 for correctly stating \uline{all} eigenvalues, 5 for having all of the above]}
\end{solution}


\question[25] Briefly explain why the matrix 
\begin{equation*}
A=
\begin{bmatrix}
1&2&3\\
0&4&5\\
0&0&6
\end{bmatrix}
\end{equation*}
is similar to a diagonal matrix.
\addpoints

\begin{solution}
	By a theorem introduced in class, all triangular matrices have eigenvalues on its diagonal entries. As such, $A$ is diagonalisable: for each eigenvalue (in this case, $1,4,6$), one can find a corresponding nonzero vector (eigenvector) such that $Ax = \lambda x$. By principles of diagonalisation, since there are $n$ many distinct eigenvalues and $n$ corresponding distinct eigenvectors, we can construct $P$ (with eigenvectors as its columns) and $D$ (with eigenvalues on its diagonal) such that $A = PDP^{-1}$, hence $A$ is conjugate of $D$ by definition.
	
	\textit{[5 for stating all triangular matrices have eigenvalues (possibly including the eigenvalues in this case), 5 for attempting to state what diagonalisation is, 4 for description of $P$, 4 for description of $D$, 4 for justification using the definition of conjugates, 3 for having all of the above]}
\end{solution}

\end{questions}


\clearpage


\section*{General Comments}

The following are general observations from the responses collected.


\begin{enumerate}
	\item \begin{enumerate}
		\item The most glaring flaw in attempts to define \textit{eigenvector} is the failure to exclude the case when $Av = \lambda v$ trivially true: when $v = 0$. We are not interested in looking for the set of values $\lambda$ such that the relation holds, since the statement would be true for all $\lambda \in \R$.
		
		An eigenvector is associated with an eigenvalue, counting multiplicities.
		
		\item Qualifiers are important to state in the definition; just stating $Q^{-1}$ without stating $Q$ is invertible is \textit{not} technically wrong, but then $Q^{-1}$ is just a symbol.
	\end{enumerate}


	\item For some reason, people use the power rule for diagonal matrices when looking at $U$, an upper triangular matrix. Be careful with calculations.
	
	For the two or three people who got \textit{everything} but $A^{10}$, \textit{please be careful with matrix multiplications}.
	
	
	\item Writing out $\det(A-\lambda I)$ fully gives
	\begin{align*}
	\det \begin{bmatrix}
	1-\lambda & & \\ & 1-\lambda & 1 \\ & 1 & 1-\lambda
	\end{bmatrix}
	\end{align*}
	so the polynomial is $(1-\lambda)^3 - (1-\lambda)$. Be careful when setting up the determinant.
	
	
	\item There were some \textit{weird} responses to this. Some did state the eigenvalues correctly, but fail to use definition of conjugates to explain when $A \sim D$, where $\sim$ denotes the conjugate relation and $D$ is a $3 \times 3$ diagonal matrix. Others tried to say a change of coordinates is possible to transform $A$ into a diagonal matrix. The latter is the basic heuristics, but spelling it out is important.
	
	As Alekos tried to explain in the notes posted last week, the main point of diagonalisation is to be able to say $A$ and $D$ represent the same linear transformation, $T(v) = \lambda v$, where a nonzero $v$ is called eigenvector and corresponding $\lambda \in \R$ is the eigenvalue. Conjugation by change of basis matrix $P$ gives us this relation.
	
	Also, just because a matrix has independent columns \textit{does not mean} it is diagonalisable. A basic counterexample is the rotation matrix in $\R^2$:
	\begin{align*}
	R = \begin{bmatrix}
	0 & -1 \\ 1 & 0
	\end{bmatrix}
	\end{align*}
	and one can easily tell that the characteristic polynomial does not have a solution in $\R$ (it does in $\CC$, though).
	
	For those who did not explain what $D$ and $P$ are: this is problematic, because those form the reasoning for why $A \sim D$!
\end{enumerate}




\end{document}

