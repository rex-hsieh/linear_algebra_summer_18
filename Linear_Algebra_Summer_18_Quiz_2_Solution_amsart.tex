\documentclass[answers,11pt]{exam}
\usepackage[utf8]{inputenc}
\usepackage[letterpaper,top=1.4in,bottom=1.2in,right=1.3in,left=1.3in]{geometry}


\newcommand{\class}{Linear Algebra}
\newcommand{\term}{Summer 2018}
\newcommand{\examnum}{Quiz 2: Suggested Solutions}
\newcommand{\examdate}{07.16.18}
\newcommand{\timelimit}{20 Minutes}

\pagestyle{head}
\firstpageheader{}{}{}
\runningheader{\class}{\examnum\ - Page \thepage\ of \numpages}{\examdate}
\runningheadrule

\usepackage{hyphenat}
\usepackage{graphicx}
\usepackage{tikz}
\usepackage{pgfplots}
\usepackage{mathrsfs}

\usepackage{array, booktabs, caption}
\usepackage{ragged2e}
\usepackage{multirow}
\usepackage{hhline}% http://ctan.org/pkg/hhline
\usepackage{makecell} 
\usepackage{enumitem}
\usepackage{bbm}
\usepackage{soul}

%\usepackage{mtpro2}
%\usepackage[nottoc]{tocbibind}

\usepackage[bottom]{footmisc}
\usepackage{natbib}
\usepackage{hyperref}
\usepackage{url}

\usepackage[normalem]{ulem}
\usepackage{relsize}
\usepackage{commath}
\usepackage{mathtools}
\allowdisplaybreaks

\usepackage{float}

\usepackage{color}
\usepackage{commath}
\usepackage{amsthm}

\newcommand\independent{\protect\mathpalette{\protect\independenT}{\perp}}
\def\independenT#1#2{\mathrel{\rlap{$#1#2$}\mkern2mu{#1#2}}}

\usepackage[utf8]{inputenc}
\usepackage[english]{babel}
\usepackage{textcomp}

\renewcommand\theadalign{lc}
\renewcommand\theadfont{\bfseries}
\usepackage{amssymb} %maths
\usepackage{amsmath} %maths

\usepackage[utf8]{inputenc} %useful to type directly diacritic characters

\usepackage{nomencl}
\makenomenclature

\usepackage{accents}

\makeatletter
\renewcommand*\env@matrix[1][*\c@MaxMatrixCols c]{%
	\hskip -\arraycolsep
	\let\@ifnextchar\new@ifnextchar
	\array{#1}}
\makeatother


\newtheoremstyle{mytheoremstyle} % name
{\topsep}                    % Space above
{\topsep}                    % Space below
{}                   % Body font
{}                           % Indent amount
{\bfseries}                   % Theorem head font
{.}                          % Punctuation after theorem head
{.5em}                       % Space after theorem head
{}  % Theorem head spec (can be left empty, meaning ‘normal’)


\theoremstyle{definition}
\newtheorem{definition}{Definition} %[section]
\newtheorem{proposition}[definition]{Proposition}
\newtheorem{theorem}[definition]{Theorem}
\newtheorem{corollary}[definition]{Corollary}
\newtheorem{lemma}[definition]{Lemma}
\newtheorem{obs}[definition]{Observation}
\newtheorem{example}[definition]{Example}
\newtheorem{assumption}[definition]{Assumption}
\newtheorem{properties}[definition]{Properties}
\newtheorem{motivation}[definition]{Motivation}
\newtheorem{derivation}[definition]{Derivation}
\newtheorem{remark}[definition]{Remark}
\newtheorem{fact}[definition]{Fact}

\usepackage{caption}
\captionsetup[figure]{labelfont=sc}
\setlist[enumerate]{font=\bfseries\sffamily}



%\addto\captionsenglish{\renewcommand*{\proofname}{\scshape Proof.}}


%\numberwithin{equation}{section}

\DeclareMathOperator{\N}{\mathbb{N}}
\DeclareMathOperator{\Q}{\mathbb{Q}}
\DeclareMathOperator{\R}{\mathbb{R}}
\DeclareMathOperator{\Z}{\mathbb{Z}}
\DeclareMathOperator{\NN}{\mathcal{N}}
\DeclareMathOperator{\bindist}{\mathsf{B}}
\DeclareMathOperator{\DD}{D}
\DeclareMathOperator{\betadist}{Beta}
\DeclareMathOperator{\E}{\mathbf{E}}
\DeclareMathOperator{\cov}{Cov}
\DeclareMathOperator{\var}{Var}
\DeclareMathOperator{\samplespace}{\mathcal{S}}
\DeclareMathOperator{\suchthat}{\text{ s.t. }}
\DeclareMathOperator{\summod}{\accentset{\circ}{+}}
\DeclareMathOperator{\im}{im}
\DeclareMathOperator{\1}{\mathbbm{1}}
\DeclareMathOperator{\LL}{\mathscr{L}}
\DeclareMathOperator{\Imaginary}{Im}
\DeclareMathOperator{\supp}{Supp}
\DeclareMathOperator{\powerset}{\mathcal{P}}
\DeclareMathOperator{\normP}{norm}
\DeclareMathOperator{\BB}{\mathscr{B}}
\DeclareMathOperator{\contf}{\mathcal{C}}
\DeclareMathOperator{\riemannint}{\mathscr{R}}
\DeclareMathOperator{\osc}{osc}
\DeclareMathOperator{\sigmaalg}{\sigma-algebra}
\DeclareMathOperator{\MM}{\mathscr{M}}
\DeclareMathOperator{\diam}{diam}
\DeclareMathOperator{\D}{\dif}
\DeclareMathOperator{\Span}{span}
\DeclareMathOperator{\PP}{\mathbf{P}}
\DeclareMathOperator{\CC}{\mathscr{C}}
\DeclareMathOperator{\sgn}{sgn}

\DeclareMathOperator{\col}{Col}
\DeclareMathOperator{\nul}{Null}

\renewcommand{\leq}{\leqslant}
\renewcommand{\geq}{\geqslant}
\renewcommand{\epsilon}{\varepsilon}
\renewcommand{\phi}{\varphi}

\newcommand{\Tau}{\mathcal{T}}

%\newcommand{\vect}[1][2]{\LL(#1,#2)}
\newcommand{\Lspace}[4]{\mathscr{L}^{#1}(#2,#3,#4)}
\newcommand{\condset}[4]{\left\{ #1  : \: #2 #3 #4 \right\}}
\newcommand{\ball}[2]{B(#1,#2)}
\newcommand{\innerproduct}[2]{\left\langle #1,#2 \right\rangle}


%\titleformat{\section}
%{\centering\Large\normalfont\scshape}{\thesection .}{0.5em}{}
%
%\titleformat{\subsection}
%{\centering\large\normalfont\scshape}{\thesubsection .}{0.5em}{}

\renewcommand{\qedsymbol}{$\blacksquare$}


\begin{document}

\noindent
\begin{tabular*}{\textwidth}{l @{\extracolsep{\fill}} r @{\extracolsep{6pt}} l}
\textbf{\class} & \textbf{Name:} & \makebox[2in]{\hrulefill}\\
\textbf{\term} &&\\
\textbf{\examnum} &&\\
\textbf{\examdate} &&\\
\textbf{Time Limit: \timelimit} 
\end{tabular*}\\
\rule[2ex]{\textwidth}{2pt}

\noindent This quiz contains 2 sides (including this cover page) and 4 questions. Total of points is \numpoints.


\begin{center}
Grade Table (for grader use only)\\
\addpoints
\gradetable[v][questions]
\end{center}

\noindent
\rule[2ex]{\textwidth}{2pt}

\begin{questions}

\question[25] Let $A$ denote an $m\times n$ matrix.
\noaddpoints
\begin{parts}
\part[12.5] Define the column space of $A$, $\col(A)$. Determine the value of $k\in \mathbb{N}$ for which $\col(A)$ is a subspace of $\mathbb{R}^k$.

\begin{solution}
	The column space of $A$ is defined as
	\begin{equation}
	\label{eq1}
	\tag{$\star$}
	\begin{aligned}
		\col (A) &\coloneqq \condset{\vec{b} \in \R^m}{A \vec{x}}{=}{\vec{b} \: \: \text{ for some } \vec{x} \in \R^n} \\
		&= \condset{A\vec{x} \in \R^m}{\vec{x}}{\in}{\R^n}
	\end{aligned}
	\end{equation}
	or, in words, it is the set of linear combinations of the columns of $A$. Obviously, $k = m \in \N$ is required, since the column space is a (vector) subspace of the range of the linear transformation.
	
	\textit{[5 for \uline{correctly} stating \eqref{eq1} or equivalent, 4 for \uline{correctly} stating all vector dimensions, 3 for $k=m$]}
\end{solution}

\part[12.5] Define the null space of $A$, $\nul(A)$. Determine the value of $k\in \mathbb{N}$ for which $\nul(A)$ is a subspace of $\mathbb{R}^k$.

\begin{solution}
	The nullspace of $A$ is defined as
	\begin{align}
	\label{eq2}
	\tag{$\ast$}
		\nul (A) \coloneqq \condset{\vec{x} \in \R^n}{A \vec{x}}{=}{\vec{0}_{\R^m}}
	\end{align}
	or, in words, it is the set of vectors in $\vec{x} \in \R^n$ such that $A\vec{x} = \vec{0}_{\R^m}$ (or the solution set to $A \vec{x} = \vec{0}_{\R^m}$, where $\vec{0}_{\R^m}$ denotes the zero vector in $\R^m$). Obviously, $k = n \in \N$ is required, since the nullspace is a (vector) subspace of the domain of the linear transformation.
	
	\textit{[5 for \uline{correctly} stating \eqref{eq2} or equivalent, 4 for \uline{correctly} stating all vector dimensions, 3 for $k=n$]}
\end{solution}

\end{parts}
\addpoints

\question[25] Is the set of vectors 
\begin{equation*}
\set{ \begin{bmatrix}
1\\
1\\
-2
\end{bmatrix},
\begin{bmatrix}
-5\\
-1\\
2
\end{bmatrix},
\begin{bmatrix}
7\\
0\\
-5
\end{bmatrix} }
\end{equation*}
a basis for $\mathbb{R}^3$?
\addpoints

\begin{solution}
	We must check if the set of vectors is linearly independent. Recall a set of vectors is linearly independent iff $A \vec{x} = \vec{0}$ admits only the trivial solution, where $A$ is a matrix whose columns correspond to the aforementioned vectors. Hence, the problem boils down to checking if the nullspace of $A$ is trivial or not.
	
	Some row reductions yield
	\begin{align*}
	\begin{bmatrix}
	1 & -5 & 7 \\ 1 & -1 & 0 \\ -2 & 2 & -5
	\end{bmatrix} & \leadsto \begin{bmatrix}
	1 & -5 & 7 \\ 0 & 4 & -7 \\ 0 & -8 & 9
	\end{bmatrix} \\
	&\leadsto \begin{bmatrix}
	1 & -5 & 7 \\ 0 & 4 & -7 \\ 0 & 0 & -5
	\end{bmatrix}
	\end{align*}
	so the matrix has three column pivots, hence there is no free variable (or, seen another way, it has rank 3, hence nullspace has dimension 0). This shows that the nullspace is trivial. As such, the set of vectors is linearly independent, and any set of three linearly independent vectors spans $\R^3$ by a theorem presented from earlier in the class. By definition, this is indeed a basis.
	
	
	\textit{[5 for realising exactly what to do, 5 for attempt at row reduction, 5 for concluding that the nullspace is trivial, 5 for stating the right conclusion, 5 for having all of the above]}
\end{solution}



\question[25] Show that the matrix $A$ is invertible and find its inverse, where
\begin{equation*}
A=\begin{bmatrix}
2&4\\
-3&10
\end{bmatrix}.
\end{equation*}
\addpoints

\begin{solution}
	The easiest method is to use the fact that $\det(A) = 20 - (-12) = 32$, hence $A$ is invertible. Alternatively, one could show that $A$ row reduces to
	\begin{align*}
	\begin{bmatrix}
	2&4\\
	-3&10
	\end{bmatrix} \leadsto \begin{bmatrix}
	2&4\\
	0 &32
	\end{bmatrix} \leadsto \begin{bmatrix}
	1 &2 \\
	0 &1
	\end{bmatrix} \leadsto \begin{bmatrix}
	1 &0 \\
	0 &1
	\end{bmatrix},
	\end{align*}
	ie. it has two pivot columns, hence the columns are linearly independent. Since it is a $2\times 2$ matrix, we can find it by the formula mentioned in class; hence,
	\begin{align*}
	A^{-1} = \frac{1}{32} \begin{bmatrix}
	10 & -4 \\ 3 & 2
	\end{bmatrix}
	\end{align*}
	
	\textit{[5 for attempting to use one of the equivalent conditions of the Invertible Matrix Theorem, 5 for \uline{subsequently} stating that $A$ is invertible, 5 for stating/acknowledging the determinant formula for $2 \times 2 $ matrices, 5 for correctly finding the inverse of the matrix, 5 for having all of the above]}
\end{solution}


\question[25] Compute $\det(T)$, where 
\begin{equation*}
T=
\begin{bmatrix}
3&46&20&3\\
0&7&2&0\\
0&0&4&2\\
0&0&0&1
\end{bmatrix}.
\end{equation*}

\begin{solution}
	Determinant of upper triangular matrix is given by the product of diagonal entries, hence $\det(A) = 3 \times 7 \times 4 \times 1 = 84$.
	
	\textit{[10 for using cofactor expansion OR realising the matrix is upper triangular, 5 for attempt at calculation, 5 for correct determinant, 5 for having all of the above]}
\end{solution}


\end{questions}

\clearpage


\section*{General Comments}

The following are general observations from the responses collected.


\begin{enumerate}
	\item 
	
	
	
	\item 
	
	
	
	
	
	\item 
	
	
	
	
	\item
\end{enumerate}



\end{document}

