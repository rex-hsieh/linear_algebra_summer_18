\documentclass{amsart}
%\documentclass[11pt]{article}
%\usepackage{amscls}
\linespread{1}
%\usepackage[letterpaper,top=1.2in,bottom=1in,right=1.03in,left=1.03in]{geometry}
\usepackage[letterpaper,top=1.6in,bottom=1.3in,right=1.3in,left=1.3in]{geometry}
\usepackage{titlesec}
\usepackage{lipsum}

%\usepackage{libertine}
%\usepackage[lite,subscriptcorrection,nofontinfo,amsbb,eucal]{mtpro2}
%slantedGreek

\usepackage{hyphenat}
\usepackage{graphicx}
\usepackage{tikz}
\usepackage{pgfplots}
\usepackage{mathrsfs}



\usepackage{array, booktabs, caption}
\usepackage{ragged2e}
\usepackage{multirow}
\usepackage{hhline}% http://ctan.org/pkg/hhline
\usepackage{makecell} 
\usepackage{enumitem}
\usepackage{bbm}
\usepackage{soul}

%\usepackage{mtpro2}


%\usepackage[nottoc]{tocbibind}
\usepackage[bottom]{footmisc}
\usepackage{natbib}
\usepackage{hyperref}
\usepackage{url}

\usepackage[normalem]{ulem}
\usepackage{relsize}
\usepackage{commath}
\usepackage{mathtools}
\allowdisplaybreaks

\title{\textsc{Assignment 2 : Chapters 1 \& 2}}
\thanks{Summer 2018. Instructor: Antonios-Alexandros Robotis.}
\date{\today}


\usepackage{float}

\usepackage{color}
\usepackage{commath}
\usepackage{amsthm}

\newcommand\independent{\protect\mathpalette{\protect\independenT}{\perp}}
\def\independenT#1#2{\mathrel{\rlap{$#1#2$}\mkern2mu{#1#2}}}

\usepackage[utf8]{inputenc}
\usepackage[english]{babel}
\usepackage{textcomp}

\renewcommand\theadalign{lc}
\renewcommand\theadfont{\bfseries}
\usepackage{amssymb} %maths
\usepackage{amsmath} %maths

\usepackage[utf8]{inputenc} %useful to type directly diacritic characters

\usepackage{nomencl}
\makenomenclature

\usepackage{accents}

\makeatletter
\renewcommand*\env@matrix[1][*\c@MaxMatrixCols c]{%
	\hskip -\arraycolsep
	\let\@ifnextchar\new@ifnextchar
	\array{#1}}
\makeatother


\usepackage{fancyhdr}

\pagestyle{fancy}
\fancyhf{}
\renewcommand{\headrulewidth}{0.5pt}
\rhead{\textsc{Assignment 1}}
\lhead{\textsc{Linear Algebra}}
%\lhead{\textsc{}}
\cfoot{\thepage}

\usepackage{lmodern}

\newtheoremstyle{mytheoremstyle} % name
{\topsep}                    % Space above
{\topsep}                    % Space below
{}                   % Body font
{}                           % Indent amount
{\bfseries}                   % Theorem head font
{.}                          % Punctuation after theorem head
{.5em}                       % Space after theorem head
{}  % Theorem head spec (can be left empty, meaning ‘normal’)


%\theoremstyle{mytheoremstyle}
%
\theoremstyle{definition}
\newtheorem{definition}{Definition}%[section]
\newtheorem{proposition}[definition]{Proposition}
\newtheorem{theorem}[definition]{Theorem}
\newtheorem{corollary}[definition]{Corollary}
\newtheorem{lemma}[definition]{Lemma}
\newtheorem{obs}[definition]{Observation}
\newtheorem{example}[definition]{Example}
\newtheorem{assumption}[definition]{Assumption}
\newtheorem{properties}[definition]{Properties}
\newtheorem{motivation}[definition]{Motivation}
\newtheorem{derivation}[definition]{Derivation}
\newtheorem{remark}[definition]{Remark}
\newtheorem{fact}[definition]{Fact}

\theoremstyle{definition}
\newtheorem*{solution}{Solution}

\usepackage{caption}
\captionsetup[figure]{labelfont=sc}
\setlist[enumerate]{font=\bfseries\sffamily}



%\addto\captionsenglish{\renewcommand*{\proofname}{\scshape Proof.}}


%\numberwithin{equation}{section}

\DeclareMathOperator{\N}{\mathbb{N}}
\DeclareMathOperator{\Q}{\mathbb{Q}}
\DeclareMathOperator{\R}{\mathbb{R}}
\DeclareMathOperator{\Z}{\mathbb{Z}}
\DeclareMathOperator{\NN}{\mathcal{N}}
\DeclareMathOperator{\bindist}{\mathsf{B}}
\DeclareMathOperator{\DD}{D}
\DeclareMathOperator{\betadist}{Beta}
\DeclareMathOperator{\E}{\mathbf{E}}
\DeclareMathOperator{\cov}{Cov}
\DeclareMathOperator{\var}{Var}
\DeclareMathOperator{\samplespace}{\mathcal{S}}
\DeclareMathOperator{\suchthat}{\text{ s.t. }}
\DeclareMathOperator{\summod}{\accentset{\circ}{+}}
\DeclareMathOperator{\im}{im}
\DeclareMathOperator{\1}{\mathbbm{1}}
\DeclareMathOperator{\LL}{\mathscr{L}}
\DeclareMathOperator{\Imaginary}{Im}
\DeclareMathOperator{\supp}{Supp}
\DeclareMathOperator{\powerset}{\mathcal{P}}
\DeclareMathOperator{\normP}{norm}
\DeclareMathOperator{\BB}{\mathscr{B}}
\DeclareMathOperator{\contf}{\mathcal{C}}
\DeclareMathOperator{\riemannint}{\mathscr{R}}
\DeclareMathOperator{\osc}{osc}
\DeclareMathOperator{\sigmaalg}{\sigma-algebra}
\DeclareMathOperator{\MM}{\mathscr{M}}
\DeclareMathOperator{\diam}{diam}
\DeclareMathOperator{\D}{\dif}
\DeclareMathOperator{\Span}{span}
\DeclareMathOperator{\PP}{\mathbf{P}}
\DeclareMathOperator{\CC}{\mathscr{C}}
\DeclareMathOperator{\sgn}{sgn}

\renewcommand{\leq}{\leqslant}
\renewcommand{\geq}{\geqslant}
\renewcommand{\epsilon}{\varepsilon}
\renewcommand{\phi}{\varphi}

\newcommand{\Tau}{\mathcal{T}}

%\newcommand{\vect}[1][2]{\LL(#1,#2)}
\newcommand{\Lspace}[4]{\mathscr{L}^{#1}(#2,#3,#4)}
\newcommand{\condset}[4]{\left\{ #1  : \: #2 #3 #4 \right\}}
\newcommand{\ball}[2]{B(#1,#2)}
\newcommand{\innerproduct}[2]{\left\langle #1,#2 \right\rangle}


%\titleformat{\section}
%{\centering\Large\normalfont\scshape}{\thesection .}{0.5em}{}
%
%\titleformat{\subsection}
%{\centering\large\normalfont\scshape}{\thesubsection .}{0.5em}{}

\renewcommand{\qedsymbol}{$\blacksquare$}

\begin{document}
\sloppy
\maketitle

%In what follows, let
%\begin{itemize}[itemsep=0mm]
%	\item $V$ be an inner product space.
%\end{itemize}

This document contains suggested solutions to a selection of \textit{representative} problems (which are graded) from the current homework assignment---one from each subsection of chapter one in the textbook. They do not include every last detail, but should give you enough guidance to complete each problem fully on your own. 

Even though the rest of the homework assignment are checked for completion, the grader will make an attempt at pointing out glaring faults in arguments whenever they arise. As such, some general comments on the completion of this assignment are provided at the end. 

Below are a few conventions are used throughout grading your assignments, along with some ground rules for grading:


\begin{enumerate}[itemsep=.75em]
	
	\item Raw score is given /100, but normalised to /10 on NYUClasses. The raw score makes partial marks easier to give out.
	
	\item Up to 40 raw marks are given for completion of homework. The actual score given depends on the amount of assignment completed, and quality of the work attempted. This will be given in 5-mark increments. As a rough description for this raw score, the table below describes how this score will be assigned:
	
	\begin{center}
		\begin{tabular}{c | p{9cm}}
			\textbf{Mark(s)} & \multicolumn{1}{c}{\textbf{Description}} \\
			\hline
			0 & Did not complete anything in the assignment / did not hand it in. \\
			\hline
			10 & A lighthearted attempt at a few questions are made. \\
			\hline
			20 & Up to half of the questions have been attempted, with some effort put in. \\
			\hline
			30 & More than three-fourths of the questions have been attempted, all with considerable progress. \\
			\hline
%			40 & Missing attempts on only a few questions in the assignment. \\
%			\hline
			40 & Attempted every question assigned, and has made considerable progress in every question.
		\end{tabular}
	\end{center}
	
%	\item When grading the questions indicated below, the following scale and description are considered:
%	
%	\begin{center}
%		\begin{tabular}{c | c}
%			\textbf{Mark(s)} & \textbf{Description} \\
%			\hline
%			0 & Did not complete anything to solve the problem. \\
%			2 & A mostly incorrect attempt has been made. \\
%			4 & A somewhat correct attempt has been made. \\
%			6 & One to a few mistakes has been made. \\
%			8 & One to two mistakes has been made, with wrong calculations or conclusions. \\
%			10 & Looks great!
%		\end{tabular}
%	\end{center}
%	
%	\bigskip
%	
%	\noindent Any odd-value scores are given at the discretion of the grader.
	
	\item Marks available and breakdown are indicated at the beginning of every question, inside square brackets. 
	
	\item Naming convention: in the following section, question indexed as $1.x.y$ should be read as ``question $y$ from chapter $1.x$ in Lay's textbook''.
	
	\item If you wish to dispute grading on question(s), please hand in your assignment to the instructor on the next homework submission day---with your comments/disputes written around the question(s).
	
\end{enumerate}



\clearpage

\section*{Suggested Solutions}

\bigskip

\begin{enumerate}[itemsep = 1.5mm]
	
	\item[1.9.33]
	
\end{enumerate}


\clearpage


\section*{General Comments}

The point of studying mathematics is to be able to come up with \textit{simple, nontrivial examples}, from which one can think about the fundamental principles underlying them. Some of these T/F questions help one in doing exactly that. It is important to understand how counterexamples arise in mathematics, hence I grade them in spades in this chapter.




\end{document}