\documentclass[answers,11pt]{exam}
\usepackage[utf8]{inputenc}
\usepackage[letterpaper,top=1.4in,bottom=1.2in,right=1.3in,left=1.3in]{geometry}

\usepackage{amsmath,amssymb}
%\usepackage{commath}
%\usepackage{mathtools}
%\usepackage{multicol}

\newcommand{\class}{Linear Algebra}
\newcommand{\term}{Summer 2018}
\newcommand{\examnum}{Quiz 3: Suggested Solutions}
\newcommand{\examdate}{07.24.18}
\newcommand{\timelimit}{20 Minutes}

\pagestyle{head}
\firstpageheader{}{}{}
\runningheader{\class}{\examnum\ - Page \thepage\ of \numpages}{\examdate}
\runningheadrule

\usepackage{hyphenat}
\usepackage{graphicx}
\usepackage{tikz}
\usepackage{pgfplots}
\usepackage{mathrsfs}

\usepackage{array, booktabs, caption}
\usepackage{ragged2e}
\usepackage{multirow}
\usepackage{hhline}% http://ctan.org/pkg/hhline
\usepackage{makecell} 
\usepackage{enumitem}
\usepackage{bbm}
\usepackage{soul}

%\usepackage{mtpro2}
%\usepackage[nottoc]{tocbibind}

\usepackage[bottom]{footmisc}
\usepackage{natbib}
\usepackage{hyperref}
\usepackage{url}

\usepackage[normalem]{ulem}
\usepackage{relsize}
\usepackage{commath}
\usepackage{mathtools}
\allowdisplaybreaks

\usepackage{float}

\usepackage{color}
\usepackage{commath}
\usepackage{amsthm}

\newcommand\independent{\protect\mathpalette{\protect\independenT}{\perp}}
\def\independenT#1#2{\mathrel{\rlap{$#1#2$}\mkern2mu{#1#2}}}

\usepackage[utf8]{inputenc}
\usepackage[english]{babel}
\usepackage{textcomp}

\renewcommand\theadalign{lc}
\renewcommand\theadfont{\bfseries}
\usepackage{amssymb} %maths
\usepackage{amsmath} %maths

\usepackage[utf8]{inputenc} %useful to type directly diacritic characters

\usepackage{nomencl}
\makenomenclature

\usepackage{accents}

\makeatletter
\renewcommand*\env@matrix[1][*\c@MaxMatrixCols c]{%
	\hskip -\arraycolsep
	\let\@ifnextchar\new@ifnextchar
	\array{#1}}
\makeatother

\makeatletter
\renewcommand*\env@matrix[1][\arraystretch]{%
	\edef\arraystretch{#1}%
	\hskip -\arraycolsep
	\let\@ifnextchar\new@ifnextchar
	\array{*\c@MaxMatrixCols c}}
\makeatother


\newtheoremstyle{mytheoremstyle} % name
{\topsep}                    % Space above
{\topsep}                    % Space below
{}                   % Body font
{}                           % Indent amount
{\bfseries}                   % Theorem head font
{.}                          % Punctuation after theorem head
{.5em}                       % Space after theorem head
{}  % Theorem head spec (can be left empty, meaning ‘normal’)


\theoremstyle{definition}
\newtheorem{definition}{Definition} %[section]
\newtheorem{proposition}[definition]{Proposition}
\newtheorem{theorem}[definition]{Theorem}
\newtheorem{corollary}[definition]{Corollary}
\newtheorem{lemma}[definition]{Lemma}
\newtheorem{obs}[definition]{Observation}
\newtheorem{example}[definition]{Example}
\newtheorem{assumption}[definition]{Assumption}
\newtheorem{properties}[definition]{Properties}
\newtheorem{motivation}[definition]{Motivation}
\newtheorem{derivation}[definition]{Derivation}
\newtheorem{remark}[definition]{Remark}
\newtheorem{fact}[definition]{Fact}

\usepackage{caption}
\captionsetup[figure]{labelfont=sc}
\setlist[enumerate]{font=\bfseries\sffamily}



%\addto\captionsenglish{\renewcommand*{\proofname}{\scshape Proof.}}


%\numberwithin{equation}{section}

\DeclareMathOperator{\N}{\mathbb{N}}
\DeclareMathOperator{\Q}{\mathbb{Q}}
\DeclareMathOperator{\R}{\mathbb{R}}
\DeclareMathOperator{\Z}{\mathbb{Z}}
\DeclareMathOperator{\NN}{\mathcal{N}}
\DeclareMathOperator{\bindist}{\mathsf{B}}
\DeclareMathOperator{\DD}{D}
\DeclareMathOperator{\betadist}{Beta}
\DeclareMathOperator{\E}{\mathbf{E}}
\DeclareMathOperator{\cov}{Cov}
\DeclareMathOperator{\var}{Var}
\DeclareMathOperator{\samplespace}{\mathcal{S}}
\DeclareMathOperator{\suchthat}{\text{ s.t. }}
\DeclareMathOperator{\summod}{\accentset{\circ}{+}}
\DeclareMathOperator{\im}{im}
\DeclareMathOperator{\1}{\mathbbm{1}}
\DeclareMathOperator{\LL}{\mathscr{L}}
\DeclareMathOperator{\Imaginary}{Im}
\DeclareMathOperator{\supp}{Supp}
\DeclareMathOperator{\powerset}{\mathcal{P}}
\DeclareMathOperator{\normP}{norm}
\DeclareMathOperator{\BB}{\mathcal{B}}
\DeclareMathOperator{\contf}{\mathcal{C}}
\DeclareMathOperator{\riemannint}{\mathscr{R}}
\DeclareMathOperator{\osc}{osc}
\DeclareMathOperator{\sigmaalg}{\sigma-algebra}
\DeclareMathOperator{\MM}{\mathscr{M}}
\DeclareMathOperator{\M}{\mathcal{M}}
\DeclareMathOperator{\diam}{diam}
\DeclareMathOperator{\D}{\dif}
\DeclareMathOperator{\Span}{span}
\DeclareMathOperator{\PP}{\mathbf{P}}
\DeclareMathOperator{\CC}{\mathscr{C}}
\DeclareMathOperator{\sgn}{sgn}

\DeclareMathOperator{\col}{Col}
\DeclareMathOperator{\nul}{Null}

\renewcommand{\leq}{\leqslant}
\renewcommand{\geq}{\geqslant}
\renewcommand{\epsilon}{\varepsilon}
\renewcommand{\phi}{\varphi}

\newcommand{\Tau}{\mathcal{T}}
\newcommand{\zerov}{\mathbf{0}}

\newcommand{\uu}{\mathbf{u}}
\newcommand{\vv}{\mathbf{v}}

%\newcommand{\vect}[1][2]{\LL(#1,#2)}
\newcommand{\Lspace}[4]{\mathscr{L}^{#1}(#2,#3,#4)}
\newcommand{\condset}[4]{\left\{ #1  : \: #2 #3 #4 \right\}}
\newcommand{\ball}[2]{B(#1,#2)}
\newcommand{\innerproduct}[2]{\left\langle #1,#2 \right\rangle}
\newcommand{\polyn}[2]{\mathcal{P}_{#1}(#2)}

%\titleformat{\section}
%{\centering\Large\normalfont\scshape}{\thesection .}{0.5em}{}
%
%\titleformat{\subsection}
%{\centering\large\normalfont\scshape}{\thesubsection .}{0.5em}{}

\renewcommand{\qedsymbol}{$\blacksquare$}


\begin{document}

\noindent
\begin{tabular*}{\textwidth}{l @{\extracolsep{\fill}} r @{\extracolsep{6pt}} l}
\textbf{\class} & \textbf{Name:} & \makebox[2in]{\hrulefill}\\
\textbf{\term} &&\\
\textbf{\examnum} &&\\
\textbf{\examdate} &&\\
\textbf{Time Limit: \timelimit} 
\end{tabular*}\\
\rule[2ex]{\textwidth}{2pt}

This quiz contains 2 sides (including this cover page) and 4 questions.\\
Total of points is \numpoints.


\begin{center}
Grade Table (for grader use only)\\
\addpoints
\gradetable[v][questions]
\end{center}

\noindent
\rule[2ex]{\textwidth}{2pt}

\begin{questions}

\question[30] Define
\noaddpoints
\begin{parts}
\part[15] a subspace $H$ of a real vector space $V$.

\begin{solution}
	$H \subset V$ is a vector subspace if the following conditions hold:
	\begin{enumerate}[label=\alph*)]
		\item $\zerov_V \in H$ (the zero vector of $V$ is in $H$)
		\item $\forall \uu,\vv \in H$, $\uu + \vv \in H$ (closure under addition)
		\item $\forall \uu \in H$ and $c \in \R$ (or, a scalar), $c\uu \in H$ (closure under scalar multiplication)
	\end{enumerate}

	\textit{[3 for ``subset'', 3 for a), 3 for b), 3 for c), 3 for having all of the above. If the qualifiers are missing in each of a) through c), a maximum of 1 is given for the part]}
\end{solution}

\part[15] a linear transformation $T:V\to W$, where $V$ and $W$ are real vector spaces. 

\begin{solution}
	A linear transformation is a map between two vector spaces, $V$ and $W$, that satisfies the following properties: $\forall \uu,\vv \in V$ and $c \in \R$ (or, a scalar),
	\begin{enumerate}[label=\alph*)]
		\item $T(\uu+\vv) = T(\uu)+T(\vv)$ (additivity)
		\item $T(c\uu) = cT(\uu)$ (homogeneity)
	\end{enumerate}
	To each linear transformation we can associate a matrix representation.
	
	\textit{[4 for ``map'' or ``matrix representation'', 4 for a), 4 for b), 3 for having all of the above. If the qualifiers are missing for each of a) and b), then a maximum of 2 is given for each part]}
\end{solution}


\end{parts}
\addpoints

\question[20] Give \textbf{two} examples of $\mathbb{R}$-vector spaces, besides $\mathbb{R}^n$ for $n\in \mathbb{N}$. You do not need to prove that these are vector spaces, but be sure to describe the sets and the addition and scaling operations on them.
\addpoints

\begin{solution}
	Some (interesting) examples of $\R$-vector spaces are as follows:
	\begin{itemize}
		\item The set of polynomials of degree $\leq n$: ie.
		\begin{align*}
			\polyn{n}{\R} \coloneqq \condset{a_0 + a_1 x + a_2 x^2 + \dots + a_n x^n}{ \set{a_i}_{i=0}^{n} \in \R}{ \text{ and }}{n \in \N}
 		\end{align*}
 		where the operations addition and scalar multiplication are defined as, for all $p,q \in \polyn{n}{\R}$ and $\lambda \in \R$,
 		\begin{align*}
 		(p+q)(x) &= p(x) + q(x) \\
 		(\lambda p)(x) &= \lambda p(x)
 		\end{align*}
 		
 		\item The set of all real-valued functions: ie.
 		\begin{align*}
 			V \coloneqq \condset{f}{f}{:}{\R \mapsto \R}
 		\end{align*}
 		under the same addition and scalar multiplication definitions as above.
 		
 		
 		\item The set of real-valued functions that are solutions to a differential equation: ie.
 		\begin{align*}
 		W \coloneqq \condset{f: \R \mapsto \R}{\dpd[2]{f}{x}+f}{=}{0}
 		\end{align*}
 		under the same addition and scalar multiplication definitions as above. Because differentiation is a linear map, ie. 
 		\begin{align*}
 		 \dpd[2]{(f+g)}{x} = \dpd[2]{f}{x} + \dpd[2]{g}{x}
 		\end{align*}
 		linearity holds. Homogeneity can be argued in a similar way.
 		
 		\item The set of real-valued continuous functions: ie.
 		\begin{align*}
 		\mathcal{W} \coloneqq \condset{f \in \contf}{f}{:}{\R \mapsto \R}
 		\end{align*}
	\end{itemize}

	\textbf{N.B.} As Alekos pointed out, if you provided examples of subspaces of $\R^n$, you will receive points, even though this is \textit{not} what the question is asking.
	
	\textit{[4 for correct first example, 6 for correct descriptions of the first set and the operations associated, 4 for correct second example, 6 for correct descriptions of the second set and the operations associated, 5 for having all of the above]}
\end{solution}


\question[25] Decide whether or not 
\begin{equation*}
\BB \coloneqq \left\{
\begin{bmatrix}
1\\
0\\
-2
\end{bmatrix}, 
\begin{bmatrix}
3\\
2\\
-4
\end{bmatrix}, 
\begin{bmatrix}
-3\\
-5\\
1
\end{bmatrix}
\right\}
\end{equation*}
is a basis for $\mathbb{R}^3$. Show all of your work.
\addpoints

\begin{solution}
	Recall a basis is, by definition, a set of vectors that are linearly independent and spanning (the ambient space). From a few results from earlier, we know that asking if vectors in $\BB$ are linearly independent is equivalent to checking if $A\vec{x} = \vec{0}$ admits only the trivial solution, where 
	\begin{align*}
	A = \begin{bmatrix}
	1& 3 & -3\\
	0& 2 & -5 \\
	-2 & -4 &1
	\end{bmatrix}
	\end{align*}
	or, equivalently, if $\nul(A) = \{ \vec{0} \}$. So we need to see if the nullspace is trivial or not.
	
	A simple row reduction exercise reveals
	\begin{align*}
	\begin{bmatrix}
	1& 3 & -3\\
	0& 2 & -5 \\
	-2 & -4 &1
	\end{bmatrix} & \leadsto \begin{bmatrix}
	1& 3 & -3\\
	0& 2 & -5 \\
	0 & 2 & -5
	\end{bmatrix} \leadsto \begin{bmatrix}
	1& 3 & -3\\
	0& 2 & -5 \\
	0 & 0 & 0
	\end{bmatrix}
	\end{align*}
	And, as the number of column pivots in RREF of $A$ is the dimension of the column space, we have, by rank-nullity,
	\begin{align*}
	3 = 2 + \dim (\nul(A)) \implies \dim(\nul(A)) = 1
	\end{align*}
	which is obviously not trivial. In fact, one can explicitly solve for the nullspace, and get
	\begin{align*}
	\nul(A) = \condset{z \begin{bmatrix}[1.2]
		-\frac{9}{2} \\ \frac{5}{2} \\ 1
		\end{bmatrix}}{z}{\in}{\R}
	\end{align*}
	
	An alternative is to calculate the determinant and see that it is zero, hence the matrix $A$ (as defined above) is not invertible. As such, the columns are not linearly independent, hence the nullspace has dimension greater than 0.
	
	\textit{[5 for stating (in some way) the definition of a basis, 5 for an attempt at row reduction or determinant calculation or equivalent method, 5 for correct calculations throughout, 5 for correct explanation(s), 5 for having all of the above]}
\end{solution}


\question[25] For which values of $a,b\in \mathbb{R}$ is the map $T:\mathbb{R}\to \mathbb{R}$ given by $T(x)=ax+b$ a linear transformation.

\begin{solution}
	In order for $T$ to be a linear transformation, it must satisfy both additivity and homogeneity, ie.
	\begin{itemize}
		\item $T(x+y) = T(x) + T(y)$ for all $x,y \in \R$
		\item $T(\lambda x) = \lambda T(x)$ for all $x, \lambda \in \R$ (where $\lambda$ is some fixed constant)
	\end{itemize}
	Note that $T(x+y) = a(x+y)+b$ and $T(\lambda x) = a\lambda x + b$, whereas $T(x)+T(y) = a(x+y) +2b$ and $\lambda T(x) = a \lambda x + \lambda b$. Combining the expressions give rise to $2b = b$ and $\lambda b = b$. Since $\lambda $ can be any fixed constant, we conclude that the only possible solution is $b = 0$.
	
	Hence, for all $a \in \R$ and $b=0$, $T(x)$ is a linear transformation.
	
	The map, as defined in the question, is called an \textit{affine map}.
	
	\textit{[5 for attempting to use the definition of linear transformation, 5 for correctly stating the necessary conditions for $T$ to be a linear transformation, 5 for correct calculations, 1 for an attempt of any kind at stating $a$ and $b$, 4 for correct values of $a$ and $b$, 5 for having all of the above]}
\end{solution}



\end{questions}


\clearpage


\section*{General Comments}

\textit{[Developing...]}


The following are general observations from the responses collected.



\begin{enumerate}
	\item 
	
	
	\item 
	
	
	\item 
	
	
	
	\item  
\end{enumerate}




\end{document}

